\documentclass[12pt,a4paper]{article}
\usepackage[utf8]{inputenc}
\usepackage[T1]{fontenc}
\usepackage{amsmath}
\usepackage{amsfonts}
\usepackage{amssymb}
\usepackage{graphicx}
\usepackage[width=0.00cm, left=3.00cm, right=2.00cm, top=3.00cm, bottom=2.00cm]{geometry}
\title{Lista 6}
\date{}
\begin{document}	
\maketitle
\begin{center}
\textbf{Curso de Ciências atuariais}

\textbf{Disciplina Probabilidade 1 - Professora Cristina}

\textbf{08/08/2022  - Exercícios de variável aleatória discreta, F(X), E(X) e V(X)}
\end{center}

\vspace{1cm}
1) Uma moeda apresenta cara 3 vezes mais frequente que coroa. Essa moeda é jogada 2 vezes. Encontre a distribuição de probabilidade de X=número de caras que aparece.

a) Encontre a distribuição de probabilidade

b) Encontre a função distribuição acumulada e faça sua representação gráfica;

c) Determine os valores de E(X) e V(X)

\vspace{1cm}
2) Um lote que contem 20 peças das quais 5 são defeituosas serão retiradas 3 peças. Encontre a distribuição de probabilidade de X=número de peças defeituosas encontradas, nos seguintes casos:

a) Retirada um a um com reposição;

b) Retirada um a um sem reposição.

Em cada caso:

c) Encontre a função distribuição acumulada e faça sua representação gráfica;

d) Determine os valores de E(X) e V(X)

\vspace{1cm}
3) Duas cartas são selecionadas aleatoriamente, sem reposição, de uma caixa que contem 5 cartas numeradas: 1. 1, 2, 2, 3. Seja X=soma das duas cartas selecionadas. Encontre a distribuição de probabilidade de X.

a) Encontre a distribuição de probabilidade

b) Encontre a função distribuição acumulada e faça sua 
representação gráfica;

c) Determine os valores de E(X) e V(X)

d) Determine $\mathbb{P}(X\geq{2})$ e $\mathbb{P}(X>1)$

\vspace{1cm}
4) Um dado é lançado duas vezes. Seja X= max(a,b) e Y=a+b; onde a é o resultado do primeiro lançamento e b é o resultado do segundo lançamento. Encontre as distribuições de probabilidade de X e de Y.

Em cada caso:

a) Encontre a distribuição de probabilidade

b) Encontre a função distribuição acumulada e faça sua representação gráfica;

c) Determine os valores de E(X) e V(X)

\vspace{1cm}
5) A distribuição de probabilidade de uma variável aleatória discreta é dada por:

\begin{table}[h]
\centering
\begin{tabular}{|c|c|c|c|}\hline
x & 1 & 2 & 3\\\hline
P(X=x) & 2a & a & 4a\\\hline
\end{tabular}
\end{table}

a) Determine o valor de a

b) Calcule as seguintes probabilidades: $\mathbb{P}(0\geq{x}\geq{3})$ e $\mathbb{P}(0<x<2)$

c) Encontre a função distribuição acumulada e faça sua representação gráfica;

d) Determine os valores de E(X) e V(X).

e) Definindo-se Y=3X, Determine E(Y) e V(Y)

\vspace{1cm}
6) De um lote com 5 bolas brancas, 3 verdes e 6 azuis serão retiradas 2 bolas. Encontre a distribuição de probabilidade da variável aleatória X=número de bolas brancas retiradas e de Y=número de bolas verdes retiradas, nos seguintes casos:

a) Retirada um a um com reposição;

b) Retirada um a um sem reposição

Em cada caso:

c) Encontre a função distribuição acumulada e faça sua representação gráfica;

d) Determine os valores de E(X) e V(X)

\vspace{1cm}
7) Uma moeda não viciada é lançada 3 vezes. Encontre a distribuição de probabilidade da variável aleatória X= número de coroas que apareceram.

a) Encontre a distribuição de probabilidade

b)  Encontre a função distribuição acumulada e faça sua representação gráfica;

c) Determine os valores de E(X) e V(X)

\vspace{1cm}
8) Uma variável aleatória discreta X pode assumir os valores 5; 10 e 15. Sabendo que F(5)=0,35, F(10)= 0,70 e F(15)=1, faça a representação gráfica de F(x) e encontre os valores de E(X) e V(X).

\vspace{1cm}
9) Aos valores da v.a. X da questão 8 foi somado o número 4 definindo-se Y=4+X. Determine E(Y) e V(Y).

\vspace{1cm}
10) Para produzir um determinado componente eletrônico se gasta R\$50,00. Este componente é vendido por R\$100,00. Um lote de 25 componentes é posto a venda. Sabe-se que no lote tem apenas 2 componentes com defeito. O comprador vai inspecionar 2 componentes e comprará o lote se encontrar no máximo um componente defeituoso. Qual o lucro esperado do produtor?

\vspace{1cm}
11) A função distribuição acumulada, F(x), de uma v.a. discreta X é dada por:

F(x) = 0, x < -2

F(x)= 0,25, se $-2\leq{x}< 1$

F(x)= 0,40, se  $1\leq{x}< 3$

F(x)= 0,70, se $3\leq{x}< 5$

F(x)= 1 se $x\geq{5}$

Encontre: $\mathbb{P}(x=3)$, $\mathbb{P}(x=4)$, F(0) e F(4) 
\end{document}