\documentclass[12pt,a4paper,draft]{article}
\usepackage[utf8]{inputenc}
\usepackage[T1]{fontenc}
\usepackage{amsmath}
\usepackage{amsfonts}
\usepackage{amssymb}
\usepackage{graphicx}
\usepackage[width=0.00cm, left=3.00cm, right=2.00cm, top=3.00cm, bottom=2.00cm]{geometry}
\title{Lista 8}
\date{}
\begin{document}
		\maketitle
		\begin{center}
			\textbf{Curso de Ciências atuariais}\\
			\textbf{Disciplina Probabilidade 1 - Professora Cristina}\\
			\textbf{19/08/2022 - Exercícios distribuição binomial}
		\end{center}
	1)\\
	a) Qual a probabilidade de obter 3 números maiores que 4, em 4 lançamentos de
	um dado?
	\vspace{0.5cm}\\
	Y = Obter um número maior que 4 em lançamento de um dado		\vspace{1cm}\\
	\begin{center}
		$\mathbb{P}\left(\text{Y} > 4\right)$ = $\dfrac{1}{6} + \dfrac{1}{6}$ = $\dfrac{2}{6}$
		\vspace{1cm}\\
	\end{center}
	X $\sim \text{Bin}\left(4, \dfrac{2}{6}\right)$
	\vspace{1cm}\\
	\begin{center}
		$\mathbb{P}$(X = 3) = $\dbinom{4}{3}$ * $\left(\dfrac{2}{6}\right)^3$ * $\left(\dfrac{2}{6}\right)^1$ = $\dfrac{4!}{3!(4-3)!}$ * $\dfrac{4}{36} * \dfrac{4}{6}$ = $\dfrac{64}{216}$ = $\dfrac{8}{27}$
	\end{center}
	\vspace{1cm}
	b) Qual a probabilidade de obter 8 caras em 10 lançamento de uma moeda não viciada?
	\vspace{0.5cm}\\
	Y = Obter cara (k) no lançamento de uma moeda
	\vspace{1cm}\\
	\begin{center}
		$\mathbb{P}\left(\text{Y} = \text{k}\right)$ = $\dfrac{1}{2}$
		\vspace{1cm}\\
	\end{center}
	X $\sim \text{Bin}\left(10, \dfrac{1}{2}\right)$
	\vspace{1cm}\\
	\begin{center}
		$\mathbb{P}$(X = 8) = $\dbinom{10}{8}$ * $\left(\dfrac{1}{2}\right)^8$ * $\left(\dfrac{1}{2}\right)^2$ = $\dfrac{10!}{8!(10-8)!}$ * $\dfrac{1}{256} * \dfrac{1}{4}$ = $\dfrac{45}{1.024}$
	\end{center}
	\vspace{1cm}
	c)Suponha que a probabilidade de um casal ter um filho com cabelos loiros seja $\frac{1}{4}$. Se houverem 6 crianças na  família, qual é a probabilidade de que metade delas tenha cabelos loiros?
	\vspace{0.5cm}
	Y = Criança nascer com cabelo loiro
	\begin{center}
		\vspace{1cm}
		$\mathbb{P}\left(\text{Y} = \text{s}\right)$ = $\dfrac{1}{4}$
		\vspace{1cm}\\
	\end{center}
	X $\sim \text{Bin}\left(6, \dfrac{1}{4}\right)$
	\vspace{1cm}\\
	\begin{center}
		$\mathbb{P}$(X = 3) = $\dbinom{6}{3}$ * $\left(\dfrac{1}{4}\right)^3$ * $\left(\dfrac{3}{4}\right)^3$ = $\dfrac{6!}{3!(6-3)!}$ * $\dfrac{1}{64} * \dfrac{27}{64}$ = $\dfrac{540}{4.096}$ = $\dfrac{135}{1.096}$
	\end{center}
	\vspace{1cm}
	d) Se a probabilidade de atingir um alvo num único disparo é 0,3, qual é a probabilidade de que em 4 disparos o alvo  seja atingido no mínimo 3 vezes?
	\vspace{0.5cm}
	Y = Atingir o alvo com um disparo
	\begin{center}
		\vspace{1cm}
		$\mathbb{P}\left(\text{Y} = \text{s}\right)$ = 0,3
	\end{center}
	\vspace{1cm}
	X $\sim \text{Bin}(4, 0.3)$
	\begin{center}
		\vspace{1cm}
		$\mathbb{P}$(X = 3) = $\dbinom{4}{3}$ * $\left(0,3\right)^3$ * $\left(0,7\right)^1$ = $\dfrac{4!}{3!(4-3)!}$ * 0,027 * 0,7 = 0,0189
	\end{center}
	\vspace{1cm}
	2)\\
	a) Um inspetor de qualidade extrai uma amostra de 10 tubos aleatoriamente de uma carga\\ muito grande de tubos que se sabe que contém 20\% de tubos defeituosos. Qual é a probabilidade de  que não mais do que 2 dos tubos	extraídos sejam defeituosos?
	\vspace{0.5cm}\\
	Y = Selecionar tubo defeituoso
	\begin{center}
		\vspace{1cm}
		$\mathbb{P}\left(\text{Y} = \text{s}\right)$ = 20\%
		\vspace{1cm}\\
	\end{center}
	X $\sim \text{Bin}\left(4, 0.2\right)$
	\vspace{1cm}\\
	\begin{center}
		$\mathbb{P}$(X > 2) = 1 - $\mathbb{P}$(X $\leq$ 2) = 1 - $\left[\mathbb{P}(\text{X} = 2) + \mathbb{P}(\text{X} = 1) + \mathbb{P}(\text{X} = 0)\right]$
		\vspace{0.5cm}\\
		$\mathbb{P}$(X = 2) = $\dbinom{10}{2}$ * $\left(0,2\right)^2$ * $\left(0,8\right)^8$ = $\dfrac{10!}{2!(10-2)!}$ * 0,04 * 0,16777 = 0,3020
		\vspace{0.5cm}\\
		$\mathbb{P}$(X = 1) = $\dbinom{10}{1}$ * $\left(0,2\right)^1$ * $\left(0,8\right)^9$ = $\dfrac{10!}{1!(10-1)!}$ * 0,2 * 0,13422 = 0,2684
		\vspace{0.5cm}\\
		$\mathbb{P}$(X = 0) = $\dbinom{10}{0}$ * (0,2)$^{0}$ * (0,8)$^{10}$ = $\dfrac{10!}{0!(10-0)!}$ * 1 * 0,10737 = 0,10737
		\vspace{0.5cm}\\
		$\mathbb{P}$(X > 2) = 1 - $\left[0,3020 + 0,2684 + 0,1074\right]$ = 1 - 0,5168 = 0,4832
	\end{center}
	\vspace{1cm}
	b) Um engenheiro de inspeção extrai uma amostra de 15 itens aleatoriamente de um\\ processo de fabricação sabido produzir 85\% de itens aceitáveis. Qual a probabilidade\\ de que 10 dos itens extraídos sejam aceitáveis?
	\vspace{0.5cm}\\
	Y = Selecionar item defeituoso
	\begin{center}
		\vspace{1cm}
		$\mathbb{P}\left(\text{Y} = \text{s}\right)$ = 85\%
		\vspace{1cm}\\
	\end{center}
	X $\sim \text{Bin}(15, 0.85)$
	\vspace{1cm}\\
	\begin{center}
		$\mathbb{P}$(X = 10) = $\dbinom{15}{10}$ * (0,85)$^{10}$ * (0,15)$^5$\\
		\vspace{0.25cm}
		$\mathbb{P}$(X = 10) = $\dfrac{15!}{10!(15-10)!}$ * 0,19687 * 0,00007 = 0,00001
	\end{center}
	\vspace{1cm}
	3) A probabilidade de ocorrência de turbulência em um determinado percurso a ser feito por uma aeronave é de 0,4 em um circuito diário. Seja X o número de voos	com turbulência em um total de 7 desses voos (ou seja, uma semana de trabalho). Qual a probabilidade de que:\\
	a) Não haja turbulência em nenhum dos 7 voos?
	\vspace{0.5cm}\\
	Y = Haver turbulência no voo
	\begin{center}
		\vspace{1cm}
		$\mathbb{P}\left(\text{Y} = \text{s}\right)$ = 0,4
		\vspace{1cm}\\
	\end{center}
	X $\sim \text{Bin}(7, 0.4)$
	\vspace{1cm}\\
	\begin{center}
		$\mathbb{P}$(X = 0) = $\dbinom{7}{0}$ * (0,4)$^{0}$ * (0,6)$^7$ = $\dfrac{7!}{0!(7-0)!}$ * 1 * 0,02799 = 0,02799
	\end{center}
	\vspace{1cm}
	b) Haja turbulência em pelo menos 3 deles?
	\vspace{0.5cm}\\
	\begin{center}
		$\mathbb{P}$(X $\geq$ 3) = 1 - $\mathbb{P}$(X < 3) = 1 - $\left[\mathbb{P}(\text{X} = 2) + \mathbb{P}(\text{X} = 1) + \mathbb{P}(\text{X} = 0)\right]$
		\vspace{0.5cm}\\
		$\mathbb{P}$(X = 2) = $\dbinom{7}{2}$ * $\left(0,2\right)^2$ * $\left(0,8\right)^5$ = $\dfrac{7!}{2!(7-2)!}$ * 0,04 * 0,32768 = 0,2752
		\vspace{0.5cm}\\
		$\mathbb{P}$(X = 1) = $\dbinom{7}{1}$ * $\left(0,2\right)^1$ * $\left(0,8\right)^6$ = $\dfrac{7!}{1!(7-1)!}$ * 0,2 * 0,26214 = 0,367
		\vspace{0.5cm}\\
		$\mathbb{P}$(X = 0) = $\dbinom{7}{0}$ * (0,2)$^0$ * (0,8)$^7$ = $\dfrac{7!}{0!(7-0)!}$ * 1 * 0,20971 = 0,2097
		\vspace{0.5cm}\\
		$\mathbb{P}$(X > 2) = 1 - $\left[0,2752 + 0,367 + 0,2097\right]$ = 1 - 0,8519 = 0,1481
	\end{center}
	\vspace{1cm}
	4) O Professor Paulo ministra, de segunda a sexta feira, aulas para uma turma com 30 homens e 20 mulheres. Suponha que todos os 50 alunos estão presentes durante as cinco aulas. Durante uma dada semana, ele decide sortear um aluno por dia para ser examinado. Se X é a variável aleatória que representa o número de dias em que um homem foi selecionado, qual a função de probabilidade, a média e a variância
	de X? Considere que o mesmo aluno pode ser selecionado mais de uma vez.
	\vspace{0.5cm}\\
	Y = Selecionar um homem
	\begin{center}
		\vspace{1cm}
		$\mathbb{P}\left(\text{Y} = \text{s}\right)$ = $\dfrac{30}{50}$ = $\dfrac{3}{5}$
		\vspace{1cm}\\
	\end{center}
	X $\sim \text{Bin}\left(5, \dfrac{3}{5}\right)$
	\vspace{1cm}\\
	\begin{center}
		$\mathbb{P}$(X = x) = $\dbinom{n}{x}$ * (p)$^{x}$ * (1 - p)$^{n - x}$
		\vspace{1cm}\\
		$\mathbb{P}$(X = 0) = $\dbinom{5}{0}$ * $\left(\dfrac{3}{5}\right)^0$ * $\left(\dfrac{2}{5}\right)^5$ = $\dfrac{5!}{0!(5-0)!}$ * 1 * $\dfrac{32}{3.125}$ = 0,01024
		\vspace{0.5cm}\\
		$\mathbb{P}$(X = 1) = $\dbinom{5}{1}$ * $\left(\dfrac{3}{5}\right)^1$ * $\left(\dfrac{2}{5}\right)^4$ = $\dfrac{5!}{1!(5-1)!}$ * $\dfrac{3}{5}$ * $\dfrac{16}{625}$ = $\dfrac{240}{3.125}$ = 0,0768
		\vspace{0.5cm}\\
		$\mathbb{P}$(X = 2) = $\dbinom{5}{2}$ * $\left(\dfrac{3}{5}\right)^2$ * $\left(\dfrac{2}{5}\right)^3$ = $\dfrac{5!}{2!(5-2)!}$ * $\dfrac{9}{25}$ * $\dfrac{8}{125}$ = $\dfrac{720}{3.125}$ = 0,2304
		\vspace{1cm}\\
		$\mathbb{P}$(X = 3) = $\dbinom{5}{3}$ * $\left(\dfrac{3}{5}\right)^3$ * $\left(\dfrac{2}{5}\right)^2$ = $\dfrac{5!}{3!(5-3)!}$ * $\dfrac{27}{125}$ * $\dfrac{4}{25}$ = $\dfrac{1.080}{3.125}$ = 0,3456
		\vspace{0.5cm}\\
		$\mathbb{P}$(X = 4) = $\dbinom{5}{4}$ * $\left(\dfrac{3}{5}\right)^4$ * $\left(\dfrac{2}{5}\right)^1$ = $\dfrac{5!}{4!(5-4)!}$ * $\dfrac{81}{625}$ * $\dfrac{2}{5}$ = $\dfrac{810}{3.125}$ = 0,2592
		\vspace{0.5cm}\\
		$\mathbb{P}$(X = 5) = $\dbinom{5}{5}$ * $\left(\dfrac{3}{5}\right)^5$ * $\left(\dfrac{2}{5}\right)^0$ = $\dfrac{5!}{5!(5-5)!}$ * $\dfrac{243}{3.125}$ * 1 = $\dfrac{243}{3.125}$ = 0,0778
		\vspace{1cm}\\
		E(X) = np
		\vspace{0.5cm}\\
		E(X) = 5 * $\dfrac{3}{5}$ = 3
		\vspace{1cm}\\
		V(X) = np(1-p)
		\vspace{0.5cm}\\
		E(X) = 5 * $\dfrac{3}{5}$ * $\dfrac{2}{5}$ = $\dfrac{6}{5}$
	\end{center}
	\vspace{1cm}
	5) Em um processo de fabricação de semicondutores, três pastilhas de um lote são testadas. Cada pastilha é classificada como passa ou falha. Suponha que a probabilidade de uma pastilha passar no teste seja de 0,8 e que as pastilhas sejam independentes. Qual é a probabilidade de que todas as três pastilhas passem no teste?
	\vspace{0.5cm}\\
	Y = Pastilha passar no teste
	\begin{center}
		\vspace{1cm}
		$\mathbb{P}\left(\text{Y} = \text{s}\right)$ = 0,8
		\vspace{1cm}\\
	\end{center}
	X $\sim \text{Bin}(3, 0.8)$
	\vspace{1cm}\\
	\begin{center}
		$\mathbb{P}$(X = 3) = $\dbinom{3}{3}$ * (0,8)$^{3}$ * (0,2)$^0$ = $\dfrac{3!}{3!(3-3)!}$ * 0,512 * 1 = 0,512
	\end{center}
	\vspace{1cm}
	6) Numa fábrica, 10\% dos copos de vidro se quebram ao serem colocados em caixas	que comportam 5 copos. Escolhendo ao acaso uma caixa, determine a probabilidade de:\\
	a) Haver 3 copos quebrados\\
	\vspace{0.5cm}\\
	Y = Copo quebrar na caixa
	\begin{center}
		\vspace{1cm}
		$\mathbb{P}\left(\text{Y} = \text{s}\right)$ = 10\% = 0,1
		\vspace{1cm}\\
	\end{center}
	X $\sim \text{Bin}(5, 0.1)$
	\vspace{1cm}\\
	\begin{center}
		$\mathbb{P}$(X = 3) = $\dbinom{5}{3}$ * (0,1)$^{3}$ * (0,9)$^2$ = $\dfrac{5!}{3!(5-3)!}$ * 0,001 * 0,81 = 0,0081
	\end{center}
	\vspace{1cm}
	b) Haver algum copo quebrado
	\vspace{0.5cm}\\
	\begin{center}
		$\mathbb{P}$(X $\geq$ 1) = 1 - $\mathbb{P}$(X < 1) = 1 - $\mathbb{P}$(X = 0)
		\vspace{0.5cm}\\
		$\mathbb{P}$(X $\geq$ 1) = 1 - $\left[\dbinom{5}{0} * (0,1)^{0} * (0,9)^5\right]$ = 1 - $\left[\dfrac{5!}{0!(5-0)!} * 1 * 0,59049\right]$ = 1 - 0,5905
		\vspace{0.5cm}\\
		$\mathbb{P}$(X $\geq$ 1) = 0,4095
	\end{center}	
\end{document}