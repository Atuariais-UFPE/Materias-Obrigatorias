\documentclass[12pt,a4paper,draft]{article}
\usepackage[utf8]{inputenc}
\usepackage[T1]{fontenc}
\usepackage{amsmath}
\usepackage{amsfonts}
\usepackage{amssymb}
\usepackage{graphicx}
\usepackage[width=0.00cm, left=3.00cm, right=2.00cm, top=3.00cm, bottom=2.00cm]{geometry}
\title{Lista 11}
\usepackage{multirow}
\usepackage{booktabs}
\date{}
\begin{document}
	\maketitle
	\begin{center}
		\textbf{Curso de Ciências Atuariais}\\
		\textbf{Disciplina Probabilidade 1- Professora Cristina}\\
		\textbf{26/09/2022 - Exercícios distribuição bidimensional}
	\end{center}
	1) Um aluno faz um teste de múltipla escolha com 4 questões do tipo Verdadeiro-Falso. Suponha que o aluno esteja “chutando” todas as questões, uma vez que ele não estudou a matéria da prova. Defina as seguintes variáveis aleatórias:\\
	X$_{1}$ = número de acertos entre as duas primeiras questões da prova\\
	Y$_{1}$ = número de acertos entre as duas últimas questões da prova\\
	X$_{2}$ = número de acertos entre as três primeiras questões da prova\\
	Y$_{2}$ = número de acertos entre as três últimas questões da prova\\
	a) Construa uma tabela com o espaço amostral associado a este experimento, listando todas as possibilidades de acerto e os valores de X$_{1}$, Y$_{1}$, X$_{2}$, Y$_{2}$ e suas probabilidades.
	\vspace{1cm}\\
	\begin{center}
		\begin{tabular}{ccccc}\midrule
			 & X$_{1}$ & X$_{1}$ & X$_{2}$ & X$_{2}$\\ \midrule
			(E, E, E, E) & 0 & 0 & 0 & 0\\ \midrule
			(E, E, E, C) & 0 & 1 & 0 & 1\\ \midrule
			(E, E, C, E) & 0 & 1 & 1 & 1\\ \midrule
			(E, C, E, E) & 1 & 0 & 1 & 1\\ \midrule
			(C, E, E, E) & 1 & 0 & 1 & 0\\ \midrule
			(E, E, C, C) & 0 & 2 & 1 & 2\\ \midrule
			(E, C, E, C) & 1 & 1 & 1 & 1\\ \midrule
			(C, E, E, C) & 1 & 1 & 1 & 1\\ \midrule
			(C, E, C, E) & 1 & 1 & 2 & 1\\ \midrule
			(C, C, E, E) & 2 & 0 & 2 & 1\\ \midrule
			(E, C, C, E) & 1 & 1 & 2 & 2\\ \midrule
			(E, C, C, C) & 1 & 2 & 2 & 3\\ \midrule
			(C, E, C, C) & 1 & 2 & 2 & 2\\ \midrule
			(C, C, E, C) & 2 & 1 & 2 & 2\\ \midrule
			(C, C, C, E) & 2 & 1 & 3 & 2\\ \midrule
			(C, C, C, C) & 2 & 2 & 3 & 3\\ \midrule
		\end{tabular}
	\end{center}
	\vspace{0.5cm}
	\[
	S_{X_{1}} =
	\begin{cases}
		(E, E, \_, \_) = 0\\
		(E, C, \_, \_); (C, E, \_, \_) = 1\\
		(C, C, \_, \_) = 2\\
	\end{cases}
	\]
	\vspace{0.5cm}\\
	\[
	S_{Y_{1}} =
	\begin{cases}
		(\_, \_, E, E) = 0\\
		(\_, \_, E, C); (\_, \_, C, E) = 1\\
		(\_, \_, C, C) = 2\\
	\end{cases}
	\]
	\vspace{0.5cm}\\
	\[
	S_{X_{2}} =
	\begin{cases}
		(E, E, E, \_) = 0\\
		(E, E, C, \_); (E, C, E, \_); (C, E, E, \_) = 1\\
		(E, C, C, \_); (C, E, C, \_); (C, C, E, \_) = 2\\
		(C, C, C, \_) = 3\\
	\end{cases}
	\]
	\vspace{0.5cm}\\
	\[
	S_{Y_{2}} =
	\begin{cases}
		(\_, E, E, E) = 0\\
		(\_, E, E, C); (\_, E, C, E); (\_, C, E, E) = 1\\
		(\_, E, C, C); (\_, C, E, C); (\_, C, C, E) = 2\\
		(\_, C, C, C) = 3\\
	\end{cases}
	\]
	\vspace{1cm}
	\begin{center}
		\begin{tabular}{cc}
			Variável & Probabilidade\\ \midrule
			\multirow{3}{*}{X$_{1}$} & $\mathbb{P}$(X$_{1}$=0) = $\dbinom{2}{0} * \left(\dfrac{1}{2}\right)^{0} * \left(\dfrac{1}{2}\right)^{2} = \dfrac{1}{4}$\\ \cmidrule{2-2}
			& $\mathbb{P}$(X$_{1}$=1) = $\dbinom{2}{1} * \left(\dfrac{1}{2}\right)^{1} * \left(\dfrac{1}{2}\right)^{1} = \dfrac{2}{4}$\\ \cmidrule{2-2}
			& $\mathbb{P}$(X$_{1}$=2) = $\dbinom{2}{2} * \left(\dfrac{1}{2}\right)^{2} * \left(\dfrac{1}{2}\right)^{0} = \dfrac{1}{4}$\\ \midrule
			\multirow{3}{*}{Y$_{1}$} & $\mathbb{P}$(Y$_{1}$=0) = $\dbinom{2}{0} * \left(\dfrac{1}{2}\right)^{0} * \left(\dfrac{1}{2}\right)^{2} = \dfrac{1}{4}$\\ \cmidrule{2-2}
			& $\mathbb{P}$(Y$_{1}$=1) = $\dbinom{2}{1} * \left(\dfrac{1}{2}\right)^{1} * \left(\dfrac{1}{2}\right)^{1} = \dfrac{2}{4}$\\ \cmidrule{2-2}
			& $\mathbb{P}$(Y$_{1}$=2) = $\dbinom{2}{2} * \left(\dfrac{1}{2}\right)^{2} * \left(\dfrac{1}{2}\right)^{0} = \dfrac{1}{4}$\\ \midrule
		\end{tabular}
		\vspace{1cm}
		\begin{tabular}{cc}
			Variável & Probabilidade\\ \midrule
			\multirow{4}{*}{X$_{2}$} & $\mathbb{P}$(X$_{2}$=0) = $\dbinom{3}{0} * \left(\dfrac{1}{2}\right)^{0} * \left(\dfrac{1}{2}\right)^{3} = \dfrac{1}{8}$\\ \cmidrule{2-2}
			& $\mathbb{P}$(X$_{2}$=1) = $\dbinom{3}{1} * \left(\dfrac{1}{2}\right)^{1} * \left(\dfrac{1}{2}\right)^{2} = \dfrac{3}{8}$\\ \cmidrule{2-2}
			& $\mathbb{P}$(X$_{2}$=2) = $\dbinom{3}{2} * \left(\dfrac{1}{2}\right)^{2} * \left(\dfrac{1}{2}\right)^{1} = \dfrac{3}{8}$\\ \cmidrule{2-2}
			& $\mathbb{P}$(X$_{2}$=3) = $\dbinom{3}{3} * \left(\dfrac{1}{2}\right)^{3} * \left(\dfrac{1}{2}\right)^{0} = \dfrac{1}{8}$\\ \midrule
			\multirow{4}{*}{Y$_{2}$} & $\mathbb{P}$(Y$_{2}$=0) = $\dbinom{3}{0} * \left(\dfrac{1}{2}\right)^{0} * \left(\dfrac{1}{2}\right)^{3} = \dfrac{1}{8}$\\ \cmidrule{2-2}
			& $\mathbb{P}$(Y$_{2}$=1) = $\dbinom{3}{1} * \left(\dfrac{1}{2}\right)^{1} * \left(\dfrac{1}{2}\right)^{2} = \dfrac{3}{8}$\\ \cmidrule{2-2}
			& $\mathbb{P}$(Y$_{2}$=2) = $\dbinom{3}{2} * \left(\dfrac{1}{2}\right)^{2} * \left(\dfrac{1}{2}\right)^{1} = \dfrac{3}{8}$\\ \cmidrule{2-2}
			& $\mathbb{P}$(Y$_{2}$=3) = $\dbinom{3}{3} * \left(\dfrac{1}{2}\right)^{3} * \left(\dfrac{1}{2}\right)^{0} = \dfrac{1}{8}$\\ \midrule
		\end{tabular}`
	\end{center}
	\vspace{1cm}
	b) Construa a função de distribuição conjunta de (X$_{1}$, Y$_{1}$) com as respectivas marginais.\\
	\vspace{1cm}\\
	\begin{center}
		\begin{tabular}{ccccc}\midrule
			\multirow{2}{*}{X$_{1}$} & \multicolumn{3}{c}{Y$_{1}$} & \multirow{2}{*}{$\mathbb{P}$(X$_{1}$ = x)}\\ \cmidrule{2-4}
			& 0 & 1 & 2 & \\ \midrule
			0 & $\dfrac{1}{16}$ & $\dfrac{2}{16}$ & $\dfrac{1}{16}$ & $\dfrac{1}{4}$\\ \midrule
			1 & $\dfrac{2}{16}$ & $\dfrac{4}{16}$ & $\dfrac{2}{16}$ & $\dfrac{2}{4}$\\ \midrule
			2 & $\dfrac{1}{16}$ & $\dfrac{2}{16}$ & $\dfrac{1}{16}$ & $\dfrac{1}{4}$\\ \midrule
			$\mathbb{P}$(Y$_{1}$ = y) & $\dfrac{1}{4}$ & $\dfrac{2}{4}$ & $\dfrac{1}{4}$ & 1\\ \midrule
		\end{tabular}
	\end{center}
	\vspace{1cm}
	c) Construa a função de distribuição conjunta de (X$_{2}$, Y$_{2}$) com as respectivas marginais.
	\vspace{1cm}\\
	\begin{center}
		\begin{tabular}{cccccc}\midrule
			\multirow{2}{*}{X$_{1}$} & \multicolumn{4}{c}{Y$_{1}$} & \multirow{2}{*}{$\mathbb{P}$(X$_{1}$ = x)}\\ \cmidrule{2-5}
			& 0 & 1 & 2 & 3 & \\ \midrule
			0 & $\dfrac{1}{16}$ & $\dfrac{1}{16}$ & 0 & 0 & $\dfrac{1}{8}$\\ \midrule
			1 & $\dfrac{1}{16}$ & $\dfrac{3}{16}$ & $\dfrac{2}{16}$ & 0 & $\dfrac{3}{8}$\\ \midrule
			2 & 0 & $\dfrac{2}{16}$ & $\dfrac{3}{16}$ & $\dfrac{1}{16}$ & $\dfrac{3}{8}$\\ \midrule
			3 & 0 & 0 & $\dfrac{1}{16}$ & $\dfrac{1}{16}$ & $\dfrac{1}{8}$\\ \midrule
			$\mathbb{P}$(Y$_{1}$ = y) & $\dfrac{1}{8}$ & $\dfrac{3}{8}$ & $\dfrac{3}{8}$ & $\dfrac{1}{8}$ & 1\\ \midrule
		\end{tabular}
	\end{center}
	\vspace{1cm}
	2) Uma moeda honesta é lançada 4 vezes. Seja X o número de caras nos 2 primeiros lançamentos e seja Y o número de caras nos 3 últimos lançamentos.\\
	a) Liste todos os elementos do espaço amostral deste experimento, especificando os valores de X e Y.\\
	\begin{center}
		\begin{tabular}{ccc}\midrule
			& X & Y\\ \midrule
			(c, c, c, c) & 0 & 0\\ \midrule
			(c, c, c, k) & 0 & 0\\ \midrule
			(c, c, k, c) & 0 & 1\\ \midrule
			(c, k, c, c) & 1 & 1\\ \midrule
			(k, c, c, c) & 1 & 1\\ \midrule
			(c, c, k, k) & 0 & 1\\ \midrule
			(c, k, c, k) & 1 & 1\\ \midrule
			(k, c, c, k) & 1 & 1\\ \midrule
			(c, k, k, c) & 1 & 2\\ \midrule
			(k, k, c, c) & 2 & 2\\ \midrule
			(k, c, k, c) & 1 & 2\\ \midrule
			(c, k, k, k) & 1 & 2\\ \midrule
			(k, c, k, k) & 1 & 2\\ \midrule
			(k, k, c, k) & 2 & 2\\ \midrule
			(k, k, k, c) & 2 & 3\\ \midrule
			(k, k, k, k) & 2 & 3\\ \midrule
		\end{tabular}
	\end{center}
	\vspace{0.5cm}
	\[
	S_{X} =
	\begin{cases}
	(c, c, \_, \_) = 0\\
	(k, c, \_, \_); (c, k, \_, \_) = 1\\
	(k, k, \_, \_) = 2\\
	\end{cases}
	\]
	\vspace{0.5cm}\\
	\[
	S_{Y} =
	\begin{cases}
	(c, c, c, \_) = 0\\
	(k, c, c, \_); (c, k, c, \_) (c, c, k, \_) = 1\\
	(k, k, c, \_); (k, c, k, \_); (c, k, k, \_) = 2\\
	(k, k, k, \_) = 3\\
	\end{cases}
	\]
	\vspace{1cm}\\
	b) Construa a função de distribuição conjunta de X e Y.\\
	\vspace{1cm}
	\begin{center}
		\begin{tabular}{cccccc} \midrule
			\multirow{2}{*}{X} & \multicolumn{4}{c}{Y} & \multirow{2}{*}{$\mathbb{P}$(X=x)}\\ \cmidrule{2-5}
			& 0 & 1 & 2 & 3 & \\ \midrule
			0 & $\dfrac{1}{16}$ & $\dfrac{2}{16}$ & $\dfrac{1}{16}$ & 0 & $\dfrac{1}{4}$\\ \midrule
			1 & $\dfrac{1}{16}$ & $\dfrac{3}{16}$ & $\dfrac{3}{16}$ & $\dfrac{1}{16}$ & $\dfrac{2}{4}$\\ \midrule
			2 & 0 & $\dfrac{1}{16}$ & $\dfrac{2}{16}$ & $\dfrac{1}{16}$ & $\dfrac{1}{4}$\\ \midrule
			$\mathbb{P}$(Y=y) & $\dfrac{1}{8}$ & $\dfrac{3}{8}$ & $\dfrac{3}{8}$ & $\dfrac{1}{8}$ & 1\\ \midrule
		\end{tabular}
	\end{center}
	\vspace{1cm}
	c) Encontre a distribuição condicional de X dado Y=3\\
	\vspace{1cm}
	\begin{center}
		\begin{tabular}{cc}
			X & $\mathbb{P}$(X=x | Y=3)\\ \midrule
			0 & $\dfrac{0}{\dfrac{1}{8}}$ = 0\\ \midrule
			1 & $\dfrac{\dfrac{1}{16}}{\dfrac{1}{8}} = \dfrac{1}{2}$\\ \midrule
			2 & $\dfrac{\dfrac{1}{16}}{\dfrac{1}{8}} = \dfrac{1}{2}$\\ \midrule
		\end{tabular}
	\end{center}
	\vspace{1cm}
	d) Calcule E(X), E(Y), Var(X), Var(Y)\\
	\vspace{1cm}\\
	\begin{center}
		E(X) = $0 * \dfrac{1}{4} + 1 * \dfrac{2}{4} + 2 * \dfrac{1}{4} = \dfrac{4}{4} = 1$
		\vspace{1cm}\\
		E(X$^2$) = $0^2 * \dfrac{1}{4} + 1^2 * \dfrac{2}{4} + 2^2 * \dfrac{1}{4} = \dfrac{6}{4}$
		\vspace{1cm}\\
		E(Y) = $0 * \dfrac{1}{8} + 1 * \dfrac{3}{8} + 2 * \dfrac{3}{8} + 3 * \dfrac{1}{8} = \dfrac{12}{8}$
		\vspace{1cm}\\
		E(Y$^2$) = $0^2 * \dfrac{1}{8} + 1^2 * \dfrac{3}{8} + 2^2 * \dfrac{3}{8} + 3^2 * \dfrac{1}{8} = \dfrac{24}{8} = 3$
		\vspace{1cm}\\
		V(X) = E(X$^2$) - E$^2$(X) = $\dfrac{6}{4} - 1^2 = \dfrac{2}{4} = \dfrac{1}{2}$
		\vspace{1cm}\\
		V(Y) = E(Y$^2$) - E$^2$(Y) = $3 - \left(\dfrac{6}{4}\right)^2 = \dfrac{12}{16} = \dfrac{3}{4}$
	\end{center}
	\vspace{1cm}
	3) Em uma clínica médica foram coletados dados em 150 pacientes, referentes ao último ano. Observou-se a ocorrência de infecções urinárias (U) e o número de parceiros sexuais (N). Os valores em \% encontram-se a seguir:\\
	\begin{center}
		\begin{tabular}{|l|l|l|l|l|}\hline
		\multirow{2}{*}{U} & \multicolumn{3}{l|}{Número de Parceiros} & \multirow{2}{*}{Total}\\ \cline{2-4}
		& 0 & 1 & 2+ & \\ \hline
		Sim & 10 & 20 & 45 & 75\\ \hline
		Não & 10 & 10 & 5 & 25\\ \hline
		Total & 20 & 30 & 50 & 100\\ \hline
		\end{tabular}
	\end{center}
	\vspace{0.5cm}
	Encontre todas as distribuições condicionais.
	\vspace{1cm}
	\begin{center}
		\begin{tabular}{cc} \midrule
			U & $\mathbb{P}$(U=u | N=0)\\ \midrule
			Sim & $\dfrac{\dfrac{10}{100}}{\dfrac{20}{100}} = \dfrac{1}{2}$\\ \midrule
			Não & $\dfrac{\dfrac{10}{100}}{\dfrac{20}{100}} = \dfrac{1}{2}$\\ \midrule
		\end{tabular}
		\hspace{1cm}
		\begin{tabular}{cc} \midrule
			U & $\mathbb{P}$(U=u | N=1)\\ \midrule
			Sim & $\dfrac{\dfrac{20}{100}}{\dfrac{30}{100}} = \dfrac{2}{3}$\\ \midrule
			Não & $\dfrac{\dfrac{10}{100}}{\dfrac{30}{100}} = \dfrac{1}{3}$\\ \midrule
		\end{tabular}
		\hspace{1cm}
		\begin{tabular}{cc} \midrule
			U & $\mathbb{P}$(U=u | N=2+)\\ \midrule
			Sim & $\dfrac{\dfrac{45}{100}}{\dfrac{50}{100}} = \dfrac{9}{10}$\\ \midrule
			Não & $\dfrac{\dfrac{5}{100}}{\dfrac{50}{100}} = \dfrac{1}{10}$\\ \midrule
		\end{tabular}
		\vspace{1cm}\\
		\begin{tabular}{cc} \midrule
			N & $\mathbb{P}$(N=n | U=Sim)\\ \midrule
			0 & $\dfrac{\dfrac{10}{100}}{\dfrac{75}{100}} = \dfrac{10}{75}$\\ \midrule
			1 & $\dfrac{\dfrac{20}{100}}{\dfrac{75}{100}} = \dfrac{20}{75}$\\ \midrule
			2+ & $\dfrac{\dfrac{45}{100}}{\dfrac{75}{100}} = \dfrac{45}{75}$\\ \midrule
		\end{tabular}
		\hspace{1cm}
		\begin{tabular}{cc} \midrule
			N & $\mathbb{P}$(N=n | U=Não)\\ \midrule
			0 & $\dfrac{\dfrac{10}{100}}{\dfrac{25}{100}} = \dfrac{2}{5}$\\ \midrule
			1 & $\dfrac{\dfrac{10}{100}}{\dfrac{25}{100}} = \dfrac{2}{5}$\\ \midrule
			2+ & $\dfrac{\dfrac{5}{100}}{\dfrac{25}{100}} = \dfrac{1}{5}$\\ \midrule
		\end{tabular}		
	\end{center}
\end{document}