\documentclass[12pt,a4paper]{article}
\usepackage[utf8]{inputenc}
\usepackage[T1]{fontenc}
\usepackage{amsmath}
\usepackage{amsfonts}
\usepackage{amssymb}
\usepackage{graphicx}
\usepackage{pgfplots}
\pgfplotsset{width=15cm}
\usepackage[width=0.00cm, left=3.00cm, right=2.00cm, top=3.00cm, bottom=2.00cm]{geometry}
\title{Lista 6}
\date{}
\begin{document}	
	\maketitle
	\begin{center}
		\textbf{Curso de Ciências atuariais}\\
		\textbf{Disciplina Probabilidade 1 - Professora Cristina}\\
		\textbf{08/08/2022  - Exercícios de variável aleatória discreta, 	F(X), E(X) e V(X)}
	\end{center}
	\vspace{1cm}
	1) Uma moeda apresenta cara 3 vezes mais frequente que coroa. Essa moeda é jogada 2 vezes. Encontre a distribuição de probabilidade de X=número de caras que aparece.\\
	a) Encontre a distribuição de probabilidade

\begin{center}
\vspace{0.25cm}
$\mathbb{P}(X=k) = 3\mathbb{P}(X=c)$

\vspace{1cm}
$\mathbb{P}(X=k) + \mathbb{P}(X=c) = 1$

$3\mathbb{P}(X=c) + \mathbb{P}(X=c) = 1$

$4\mathbb{P}(X=c) = 1$

$\mathbb{P}(X=c) = \dfrac{1}{4}$

\[
S_{x} =
\begin{cases}
(c,c) = 0 \\
(c,k);(k,c) = 1\\
(k,k) = 2
\end{cases}
\]

\vspace{0.5cm}
$\mathbb{P}(X=0) = \dfrac{1}{4}*\dfrac{1}{4}$ = $\dfrac{1}{16}$

\vspace{0.5cm}
$\mathbb{P}(X=1) = \dfrac{1}{4}*\dfrac{3}{4} + \dfrac{3}{4}*\dfrac{1}{4}$ = $\dfrac{3}{16} + \dfrac{3}{16}$ = $\dfrac{6}{16}$

\vspace{0.5cm}
$\mathbb{P}(X=2) = \dfrac{9}{4}*\dfrac{9}{4}$ = $\dfrac{9}{16}$
\end{center}

\vspace{1cm}
b) Encontre a função distribuição acumulada e faça sua representação gráfica;

\begin{center}
\vspace{0.25cm}
$\mathbb{P}(X\leq{0}) = \mathbb{P}(X=0) = \dfrac{1}{16}$

\vspace{0.5cm}
$\mathbb{P}(X\leq1) = \mathbb{P}(X\leq{0}) + \mathbb{P}(X=1) = \dfrac{1}{16} + \dfrac{6}{16} = \frac{7}{16}$

\vspace{0.5cm}
$\mathbb{P}(X\leq{2}) = \mathbb{P}(X\leq1) + \mathbb{P}(X=2) = \dfrac{7}{16} + \dfrac{9}{16} = \dfrac{16}{16} = 1$

\[
F(x) =
\begin{cases}
0, 0 < x \\
\frac{1}{16}, 0\leq{x} < 1 \\
\frac{7}{16}, 1\leq{x} < 2\\
1,x\geq{2}
\end{cases}
\]

\vspace{1cm}
	\begin{tikzpicture}
	\begin{axis}[xmin=-2,xmax=4, ymin=-1, ymax=1.2, axis lines=center, xlabel=x, ylabel=F(x)]
	\addplot[color=red, domain=-2:0]{0};
	\addplot[color=red, domain=0:1]{1/16};
	\addplot[color=red, domain=1:2]{7/16};
	\addplot[color=red, domain=2:4]{1};
	\end{axis}
	\end{tikzpicture}

\end{center}

\vspace{1cm}
c) Determine os valores de E(X) e V(X)

\begin{center}
\vspace{0.25cm}
$E(X) = \sum_{i}^{n} x_{i}*\mathbb{P}(X = x_{i})$

\vspace{1cm}
$E(X) = 0*\dfrac{1}{16}+1*\dfrac{6}{16}+2*\dfrac{9}{16} = 0 + \dfrac{6}{16} + \dfrac{18}{16} = \dfrac{24}{16}$

\vspace{1cm}
$E(X^{2}) = 0^{2}*\dfrac{1}{16}+1^{2}*\dfrac{6}{16}+2^{2}*\dfrac{9}{16} = 0 + \dfrac{6}{16} + \dfrac{36}{16} = \dfrac{42}{16}$

\vspace{1cm}
$V(X) = E(x^{2}) - E^{2}(x)$

\vspace{1cm}
$V(X) = \dfrac{42}{16}-(\dfrac{24}{16})^{2} = \dfrac{672}{256}-\dfrac{576}{256}=\dfrac{96}{256}$
\end{center}

\vspace{1cm}
2) Um lote que contem 20 peças das quais 5 são defeituosas serão retiradas 3 peças. Encontre a distribuição de probabilidade de X=número de peças defeituosas encontradas, nos seguintes casos:

a) Retirada um a um com reposição
\begin{center}
\[
S_{x} =
\begin{cases}
(P,P,P) = 0 \\
(P,P,D);(P,D,P);(D,P,P) = 1\\
(P,D,D);(D,P,D);(D,D,P) = 2\\
(D,D,D) = 3
\end{cases}
\]

\vspace{0.5cm}
$\mathbb{P}(X=0) = \dfrac{15}{20}*\dfrac{15}{20}$*$\dfrac{15}{20}$ = $\dfrac{3.375}{8.000}$

\vspace{0.5cm}
$\mathbb{P}(X=1) = \dfrac{15}{20}*\dfrac{15}{20}*\dfrac{5}{20}*3 = \dfrac{1.125}{8000}*3 = \dfrac{3.375}{8.000}$

\vspace{0.5cm}
$\mathbb{P}(X=2) = \dfrac{15}{20}*\dfrac{5}{20}*\dfrac{5}{20}*3 = \dfrac{375}{8.000}*3 = \dfrac{1.125}{8.000}$

\vspace{0.5cm}
$\mathbb{P}(X=3) = \dfrac{5}{20}*\dfrac{5}{20}*\dfrac{5}{20} = \dfrac{125}{8.000}$
\end{center}

\vspace{1cm}
b) Retirada um a um sem reposição
\[
S_{x} =
\begin{cases}
(P,P,P) = 0 \\
(P,P,D);(P,D,P);(D,P,P) = 1\\
(P,D,D);(D,P,D);(D,D,P) = 2\\
(D,D,D) = 3
\end{cases}
\]
\begin{center}
\vspace{0.5cm}
$\mathbb{P}(X=0) = \dfrac{15}{20}*\dfrac{14}{19}$*$\dfrac{13}{18}$ = $\dfrac{2.730}{6.840}$

\vspace{0.5cm}
$\mathbb{P}(X=1) = \dfrac{15}{20}*\dfrac{14}{19}*\dfrac{5}{18}*3 = \dfrac{1.050}{6.840}*3 = \dfrac{3.150}{6.840}$

\vspace{0.5cm}
$\mathbb{P}(X=2) = \dfrac{15}{20}*\dfrac{5}{19}*\dfrac{4}{18}*3 = \dfrac{300}{6.840}*3 = \dfrac{900}{6.840}$

\vspace{0.5cm}
$\mathbb{P}(X=3) = \dfrac{5}{20}*\dfrac{4}{19}*\dfrac{3}{18} = \dfrac{60}{6.840}$
\end{center}

\vspace{1cm}
Em cada caso:

c) Encontre a função distribuição acumulada e faça sua representação gráfica

\vspace{0.25cm}
COM REPOSIÇÃO
\begin{center}
	\vspace{0.25cm}
	$\mathbb{P}(X\leq{0}) = \mathbb{P}(X=0) = \dfrac{3.375}{8.000}$
	
	\vspace{0.5cm}
	$\mathbb{P}(X\leq1) = \mathbb{P}(X\leq{0}) + \mathbb{P}(X=1) = \dfrac{3.375}{8.000} + \dfrac{3.375}{8.000} = \dfrac{6.750}{8.000}$
	
	\vspace{0.5cm}
	$\mathbb{P}(X\leq{2}) = \mathbb{P}(X\leq1) + \mathbb{P}(X=2) = \dfrac{6.750}{8.000} + \dfrac{1.125}{8.000} = \dfrac{7.875}{8.000}$
	
	\vspace{0.5cm}
	$\mathbb{P}(X\leq{3}) = \mathbb{P}(X\leq{2}) + \mathbb{P}(X=3) = \dfrac{7.875}{8.000} + \dfrac{125}{8.000} = \dfrac{8.000}{8.000} = 1$
	
	\[
	F(x) =
	\begin{cases}
	0, 0 < x \\
	\frac{3.375}{8.000}, 0\leq{x} < 1 \\
	\frac{6.750}{8.000}, 1\leq{x} < 2\\
	\frac{7.875}{8.000}, 2\leq{x} < 3\\
	1,x\geq{3}
	\end{cases}
	\]
	
	\vspace{1cm}
	\begin{tikzpicture}
	\begin{axis}[xmin=-2,xmax=5, ymin=-1, ymax=1.2, axis lines=center, xlabel=x, ylabel=F(x)]
	\addplot[color=red, domain=-2:0]{0};
	\addplot[color=red, domain=0:1]{3.375/8.000};
	\addplot[color=red, domain=1:2]{6.750/8.000};
	\addplot[color=red, domain=2:3]{7.875/8.000};
	\addplot[color=red, domain=3:5]{1};
	\end{axis}
	\end{tikzpicture}
	
\end{center}

\vspace{1cm}
SEM REPOSIÇÃO
\begin{center}
	\vspace{0.25cm}
	$\mathbb{P}(X\leq{0}) = \mathbb{P}(X=0) = \dfrac{2.730}{6.840}$
	
	\vspace{0.5cm}
	$\mathbb{P}(X\leq1) = \mathbb{P}(X\leq{0}) + \mathbb{P}(X=1) = \dfrac{2.730}{6.840} + \dfrac{3.150}{6.840} = \dfrac{5.880}{6.840}$
	
	\vspace{0.5cm}
	$\mathbb{P}(X\leq{2}) = \mathbb{P}(X\leq1) + \mathbb{P}(X=2) = \dfrac{5.880}{6.840} + \dfrac{900}{6.840} = \dfrac{6.780}{6.840}$
	
	\vspace{0.5cm}
	$\mathbb{P}(X\leq{3}) = \mathbb{P}(X\leq{2}) + \mathbb{P}(X=3) = \dfrac{6.780}{6.840} + \dfrac{60}{6.840} = \dfrac{6.840}{6.840} = 1$
	
	\[
	F(x) =
	\begin{cases}
	0, 0 < x \\
	\frac{2.730}{6.840}, 0\leq{x} < 1 \\
	\frac{5.880}{6.840}, 1\leq{x} < 2\\
	\frac{6.780}{6.840}, 2\leq{x} < 3\\
	1,x\geq{3}
	\end{cases}
	\]
	
	\vspace{1cm}
	\begin{tikzpicture}
	\begin{axis}[xmin=-2,xmax=5, ymin=-1, ymax=1.2, axis lines=center, xlabel=x, ylabel=F(x)]
	\addplot[color=red, domain=-2:0]{0};
	\addplot[color=red, domain=0:1]{2.730/6.8400};
	\addplot[color=red, domain=1:2]{5.880/6.840};
	\addplot[color=red, domain=2:3]{6.840/6.840};
	\addplot[color=red, domain=3:5]{1};
	\end{axis}
	\end{tikzpicture}
	
\end{center}

d) Determine os valores de E(X) e V(X)

\vspace{0.25cm}
COM REPOSIÇÃO
\begin{center}
	\vspace{0.25cm}
	$E(X) = \sum_{i}^{n} x_{i}*\mathbb{P}(X = x_{i})$
	
	\vspace{1cm}
	$E(X) = 0 * \dfrac{3.375}{8.000}+1 * \dfrac{3.375}{8.000}+2 * \dfrac{1.125}{8.000} + 3*\dfrac{125}{8.000}= 0 + \dfrac{3.375}{8.000} + \dfrac{2.250}{8.000} + \dfrac{375}{8.000} = \dfrac{6.000}{8.000} = \dfrac{3}{4}$
	
	\vspace{0.5cm}
	$E(X^{2}) = 0^{2} * \dfrac{3.375}{8.000}+1^{2} * \dfrac{3.375}{8.000}+2^{2} * \dfrac{1.125}{8.000} + 3^{2} * \dfrac{125}{8.000} = 0 + \dfrac{3.375}{8.000} + \dfrac{4.500}{8.000} + \dfrac{1.125}{8.000}$
	\vspace{0.25cm}\\ 
	$E(X^{2}) = \dfrac{9.000}{8.000} = \dfrac{9}{8}$
	
	\vspace{1cm}
	$V(X) = E(X^{2}) - E^{2}(X)$
	
	\vspace{1cm}
	$V(X) = \dfrac{9}{8}-\left(\dfrac{3}{4}\right)^{2} = \dfrac{18}{16}-\frac{9}{16}=\dfrac{9}{16}$
\end{center}

\vspace{1cm}
SEM REPOSIÇÃO
\begin{center}
	\vspace{0.25cm}
	$E(X) = \sum_{i}^{n} x_{i}*\mathbb{P}(X = x_{i})$
	
	\vspace{1cm}
	$E(X) = 0*\dfrac{2.730}{6.840}+1*\dfrac{3.150}{6.840}+2*\dfrac{900}{6.840} + 3*\dfrac{60}{6.840} = 0 + \dfrac{3.150}{6.840} + \dfrac{1.800}{6.840} + \dfrac{180}{6.840} = \dfrac{5.130}{6.840} = \dfrac{57}{76}$
	
	\vspace{1cm}
	$E(X^{2}) = 0^{2}*\dfrac{2.730}{6.840}+1^{2}*\dfrac{3.150}{6.840}+2^{2}*\dfrac{900}{6.840} + 3^{2}*\dfrac{60}{6.840} = 0 + \dfrac{3.150}{6.840} + \dfrac{3.600}{6.840} + \dfrac{540}{6.840}$
	\vspace{0.25cm}\\
	$E(X^{2}) = \dfrac{7.290}{6.840} = \dfrac{81}{76}$
	
	\vspace{1cm}
	$V(X) = E(X^{2}) - E^{2}(X)$
	
	\vspace{1cm}
	$V(X) = \dfrac{81}{76}-(\dfrac{57}{76})^{2} = \dfrac{6.156}{5.776}-\dfrac{3.249}{5.776}=\dfrac{2.907}{50.776}$
\end{center}

\vspace{1cm}
3) Duas cartas são selecionadas aleatoriamente, sem reposição, de uma caixa que contem 5 cartas numeradas: 1, 1, 2, 2, 3. Seja X=soma das duas cartas selecionadas. Encontre a distribuição de probabilidade de X.

a) Encontre a distribuição de probabilidade
\[
S_{x} =
\begin{cases}
(1,1) = 2 \\
(1,2);(2,1) = 3\\
(1,3);(3,1);(2,2) = 4\\
(2,3);(3,2) = 5
\end{cases}
\]
\begin{center}
	\vspace{0.5cm}
	$\mathbb{P}(X=2) = \dfrac{2}{5}*\dfrac{1}{4} = \dfrac{2}{20}$
	
	\vspace{0.5cm}
	$\mathbb{P}(X=3) = \dfrac{2}{5}*\dfrac{2}{4}+\dfrac{2}{5}*\dfrac{2}{4} = \dfrac{4}{20} + \dfrac{4}{20} = \dfrac{8}{20}$
	
	\vspace{0.5cm}
	$\mathbb{P}(X=4) = \dfrac{2}{5}*\dfrac{1}{4}+\dfrac{1}{5}*\dfrac{2}{4}+\dfrac{2}{5}*\dfrac{1}{4} = \dfrac{2}{20} + \dfrac{2}{20} + \dfrac{2}{20} = \dfrac{6}{20}$
	
	\vspace{0.5cm}
	$\mathbb{P}(X=5) = \dfrac{2}{5}*\dfrac{1}{4}+\dfrac{1}{5}*\dfrac{2}{4} = \dfrac{2}{20} + \dfrac{2}{20} = \dfrac{4}{20}$
\end{center}

\vspace{1cm}
b) Encontre a função distribuição acumulada e faça sua 
representação gráfica

\begin{center}
	\vspace{0.25cm}
	$\mathbb{P}(X\leq{0}) = \mathbb{P}(X=0) = \dfrac{2}{20}$
	
	\vspace{0.5cm}
	$\mathbb{P}(X\leq1) = \mathbb{P}(X\leq{0}) + \mathbb{P}(X=1) = \dfrac{2}{20} + \dfrac{8}{20} = \dfrac{10}{20}$
	
	\vspace{0.5cm}
	$\mathbb{P}(X\leq{2}) = \mathbb{P}(X\leq1) + \mathbb{P}(X=2) = \dfrac{10}{20} + \dfrac{6}{20} = \dfrac{16}{20}$
	
	\vspace{0.5cm}
	$\mathbb{P}(X\leq{3}) = \mathbb{P}(X\leq{2}) + \mathbb{P}(X=3) = \dfrac{16}{20} + \dfrac{4}{20} = \dfrac{20}{20} = 1$
	
	\[
	F(x) =
	\begin{cases}
	0, 0 < x \\
	\frac{2}{20}, 0\leq{x} < 1 \\
	\frac{10}{20}, 1\leq{x} < 2\\
	\frac{16}{20}, 2\leq{x} < 3\\
	1,x\geq{3}
	\end{cases}
	\]
	
	\vspace{1cm}
	\begin{tikzpicture}
	\begin{axis}[xmin=-2,xmax=5, ymin=-1, ymax=1.2, axis lines=center, xlabel=x, ylabel=F(x)]
	\addplot[color=red, domain=-2:0]{0};
	\addplot[color=red, domain=0:1]{2/20};
	\addplot[color=red, domain=1:2]{10/20};
	\addplot[color=red, domain=2:3]{16/20};
	\addplot[color=red, domain=3:5]{1};
	\end{axis}
	\end{tikzpicture}
\end{center}

\vspace{1cm}
c) Determine os valores de E(X) e V(X)

\begin{center}
	\vspace{0.25cm}
	$E(X) = \sum_{i}^{n} x_{i}*\mathbb{P}(X = x_{i})$
	
	\vspace{1cm}
	$E(X) = 2*\dfrac{2}{20}+3*\dfrac{8}{20}+4*\dfrac{6}{20} + 5*\dfrac{4}{20} = \dfrac{4}{20} + \dfrac{24}{20} + \dfrac{24}{20} + \dfrac{20}{20} = \dfrac{72}{20} = \dfrac{18}{5}$
	
	\vspace{1cm}
	$E(X^{2}) = 2^{2}*\dfrac{2}{20}+3^{2}*\dfrac{8}{20}+4^{2}*\dfrac{6}{20} + 5^{2}*\dfrac{4}{20} = \dfrac{8}{20} + \dfrac{72}{20} + \dfrac{96}{20} + \dfrac{100}{20} = \dfrac{276}{20} = \dfrac{69}{5}$
	
	\vspace{1cm}
	$V(X) = E(X^{2}) - E^{2}(X)$
	
	\vspace{1cm}
	$V(X) = \dfrac{69}{5}-(\dfrac{18}{5})^{2} = \dfrac{345}{25}-\dfrac{324}{25} = \dfrac{21}{25}$
\end{center}

\vspace{1cm}
d) Determine $\mathbb{P}(X\geq{2})$ e $\mathbb{P}(X>1)$
\vspace{0.25cm}
\begin{center}
	$\mathbb{P}(X\geq{2}) = \mathbb{P}(X = 2) + \mathbb{P}(X = 3)+\mathbb{P}(X = 4) + \mathbb{P}(X = 5)$ 
	
	$\mathbb{P}(X\geq{2}) = \dfrac{2}{20}+\dfrac{8}{20}+\dfrac{6}{20}+\dfrac{4}{20} = 1$
	
	\vspace{1cm}
	$\mathbb{P}(X>1) = \mathbb{P}(X\geq{2}) = 1$
\end{center}

\vspace{1cm}
4) Um dado é lançado duas vezes. Seja X= max(a,b) e Y=a+b; onde a é o resultado do primeiro lançamento e b é o resultado do segundo lançamento. Encontre as distribuições de probabilidade de X e de Y.

Em cada caso:

a) Encontre a distribuição de probabilidade
\[
S_{x} =
\begin{cases}
(1,1) = 1 \\
(1,2);(2,1);(2,2) = 2\\
(1,3);(3,1);(2,3);(3,2);(3,3) = 3\\
(1,4);(4,1);(2,4);(4,2);(3,4);(4,3);(4,4) = 4\\
(1,5);(5,1);(2,5);(5,2);(3,5);(5,3);(4,5);(5,4);(5,5) = 5\\
(1,6);(6,1);(2,6);(6;2);(3,6);(6,3);(4;6);(6,4);(5;6);(6;5);(6;6) = 6
\end{cases}
\]
\begin{center}
	\vspace{0.5cm}
	$\mathbb{P}(X=1) = \dfrac{1}{36}$
	
	\vspace{0.5cm}
	$\mathbb{P}(X=2) = 3*\dfrac{1}{36} = \dfrac{3}{36}$
	
	\vspace{0.5cm}
	$\mathbb{P}(X=3) = 5*\dfrac{1}{36} = \dfrac{5}{36}$
	
	\vspace{0.5cm}
	$\mathbb{P}(X=4) = 7*\dfrac{1}{36} = \dfrac{7}{36}$
	
	\vspace{0.5cm}
	$\mathbb{P}(X=5) = 9*\dfrac{1}{36} = \dfrac{9}{36}$
	
	\vspace{0.5cm}
	$\mathbb{P}(X=6) = 11*\dfrac{1}{36} = \dfrac{11}{36}$
\end{center}

\[
S_{y} =
\begin{cases}
(1,1) = 2 \\
(1,2);(2,1) = 3\\
(1,3);(3,1);(2,2) = 4\\
(1,4);(4,1);(2,3);(3,2) = 5\\
(1,5);(5,1);(2,4);(4,2);(3,3) = 6\\
(1,6);(6,1);(2,5);(5;2);(3,4);(4,3) = 7\\
(2,6);(6,2);(3,5);(5,3);(4,4) = 8\\
(3,6);(6,3);(4,5);(5,4) = 9\\
(4,6);(6,4);(5,5) = 10\\
(5,6);(6,5) = 11\\
(6,6) = 12
\end{cases}
\]
\begin{center}
	\vspace{0.5cm}
	$\mathbb{P}(Y=2) = \dfrac{1}{36}$
	
	\vspace{0.5cm}
	$\mathbb{P}(Y=3) = 2*\dfrac{1}{36} = \dfrac{2}{36}$
	
	\vspace{0.5cm}
	$\mathbb{P}(Y=4) = 3*\dfrac{1}{36} = \dfrac{3}{36}$
	
	\vspace{0.5cm}
	$\mathbb{P}(Y=5) = 4*\dfrac{1}{36} = \dfrac{4}{36}$
	
	\vspace{0.5cm}
	$\mathbb{P}(Y=6) = 5*\dfrac{1}{36} = \dfrac{5}{36}$
	
	\vspace{0.5cm}
	$\mathbb{P}(Y=7) = 6*\dfrac{1}{36} = \dfrac{6}{36}$
	
	\vspace{0.5cm}
	$\mathbb{P}(Y=8) = 5*\dfrac{1}{36} = \dfrac{5}{36}$
	
	\vspace{0.5cm}
	$\mathbb{P}(Y=9) = 4*\dfrac{1}{36} = \dfrac{4}{36}$
	
	\vspace{0.5cm}
	$\mathbb{P}(Y=10) = 3*\dfrac{1}{36} = \dfrac{3}{36}$
	
	\vspace{0.5cm}
	$\mathbb{P}(Y=11) = 2*\dfrac{1}{36} = \dfrac{2}{36}$
	
	\vspace{0.5cm}
	$\mathbb{P}(Y=12) = \dfrac{1}{36}$
\end{center}

\vspace{1cm}
b) Encontre a função distribuição acumulada e faça sua representação gráfica

\begin{center}
	\vspace{0.25cm}
	$\mathbb{P}(X\leq{1}) = \mathbb{P}(X=1) = \dfrac{1}{36}$
	
	\vspace{0.5cm}
	$\mathbb{P}(X\leq2) = \mathbb{P}(X\leq{1}) + \mathbb{P}(X=2) = \dfrac{1}{36} + \dfrac{3}{36} = \dfrac{4}{36}$
	
	\vspace{0.5cm}
	$\mathbb{P}(X\leq{3}) = \mathbb{P}(X\leq2) + \mathbb{P}(X=3) = \dfrac{4}{36} + \dfrac{5}{36} = \dfrac{9}{36}$
	
	\vspace{0.5cm}
	$\mathbb{P}(X\leq{4}) = \mathbb{P}(X\leq{3}) + \mathbb{P}(X=4) = \dfrac{9}{36} + \dfrac{7}{36} = \dfrac{16}{36}$
	
	\vspace{0.5cm}
	$\mathbb{P}(X\leq{5}) = \mathbb{P}(X\leq{4}) + \mathbb{P}(X=5) = \dfrac{16}{36} + \dfrac{9}{36} = \dfrac{25}{36}$
	
	\vspace{0.5cm}
	$\mathbb{P}(X\leq{6}) = \mathbb{P}(X\leq{5}) + \mathbb{P}(X=6) = \dfrac{25}{36} + \dfrac{11}{36} = \dfrac{36}{36} = 1$
	\[
	F(x) =
	\begin{cases}
	0, 0 < x \\
	\frac{1}{36}, 0\leq{x} < 1 \\
	\frac{4}{36}, 1\leq{x} < 2\\
	\frac{9}{36}, 2\leq{x} < 3\\
	\frac{16}{36}, 3\leq{x} < 4\\
	\frac{25}{36}, 4\leq{x} < 5\\
	1,x\geq{6}
	\end{cases}
	\]
	
	\vspace{1cm}
	\begin{tikzpicture}
	\begin{axis}[xmin=-2,xmax=7, ymin=-1, ymax=1.2, axis lines=center, xlabel=x, ylabel=F(x)]
	\addplot[color=red, domain=-2:0]{0};
	\addplot[color=red, domain=0:1]{1/36};
	\addplot[color=red, domain=1:2]{4/36};
	\addplot[color=red, domain=2:3]{9/36};
	\addplot[color=red, domain=3:4]{16/36};
	\addplot[color=red, domain=4:5]{25/36};
	\addplot[color=red, domain=5:7]{1};
	\end{axis}
	\end{tikzpicture}
\end{center}

\vspace{1cm}
\begin{center}
	\vspace{0.25cm}
	$\mathbb{P}(Y\leq{2}) = \mathbb{P}(Y=1) = \dfrac{1}{36}$
	
	\vspace{0.5cm}
	$\mathbb{P}(Y\leq3) = \mathbb{P}(Y\leq{2}) + \mathbb{P}(X=3) = \dfrac{1}{36} + \dfrac{2}{36} = \dfrac{3}{36}$
	
	\vspace{0.5cm}
	$\mathbb{P}(Y\leq{4}) = \mathbb{P}(Y\leq3) + \mathbb{P}(Y=4) = \dfrac{3}{36} + \dfrac{3}{36} = \dfrac{6}{36}$
	
	\vspace{0.5cm}
	$\mathbb{P}(Y\leq{5}) = \mathbb{P}(Y\leq{4}) + \mathbb{P}(Y=5) = \dfrac{6}{36} + \dfrac{4}{36} = \dfrac{10}{36}$
	
	\vspace{0.5cm}
	$\mathbb{P}(Y\leq{6}) = \mathbb{P}(Y\leq{5}) + \mathbb{P}(Y=6) = \dfrac{10}{36} + \dfrac{5}{36} = \dfrac{15}{36}$
	
	\vspace{0.5cm}
	$\mathbb{P}(Y\leq{7}) = \mathbb{P}(Y\leq{6}) + \mathbb{P}(Y=7) = \dfrac{15}{36} + \dfrac{6}{36} = \dfrac{21}{36}$
	
		\vspace{0.5cm}
	$\mathbb{P}(Y\leq{8}) = \mathbb{P}(Y\leq{7}) + \mathbb{P}(Y=8) = \dfrac{21}{36} + \dfrac{5}{36} = \dfrac{26}{36}$
	
		\vspace{0.5cm}
	$\mathbb{P}(Y\leq{9}) = \mathbb{P}(Y\leq{8}) + \mathbb{P}(Y=9) = \dfrac{26}{36} + \dfrac{4}{36} = \dfrac{30}{36}$
	
		\vspace{0.5cm}
	$\mathbb{P}(Y\leq{10}) = \mathbb{P}(Y\leq{9}) + \mathbb{P}(Y=10) = \dfrac{30}{36} + \dfrac{3}{36} = \dfrac{33}{36}$
	
		\vspace{0.5cm}
	$\mathbb{P}(Y\leq{11}) = \mathbb{P}(Y\leq{10}) + \mathbb{P}(Y=11) = \dfrac{33}{36} + \dfrac{2}{36} = \dfrac{35}{36}$
	
	\vspace{0.5cm}
	$\mathbb{P}(Y\leq{12}) = \mathbb{P}(Y\leq{11}) + \mathbb{P}(Y=12) = \dfrac{35}{36} + \dfrac{1}{36} = \dfrac{36}{36} = 1$
	\[
	F(y) =
	\begin{cases}
	0, 1 < y \\
	\frac{1}{36}, 1\leq{y} < 2 \\
	\frac{3}{36}, 2\leq{y} < 3\\
	\frac{6}{36}, 3\leq{y} < 4\\
	\frac{10}{36}, 4\leq{y} < 5\\
	\frac{15}{36}, 5\leq{y} < 6\\
	\frac{21}{36}, 6\leq{y} < 7\\
	\frac{26}{36}, 7\leq{y} < 8\\
	\frac{30}{36}, 8\leq{y} < 9\\
	\frac{33}{36}, 9\leq{y} < 10\\
	\frac{35}{36}, 10\leq{y} < 11\\
	1,x\geq{11}
	\end{cases}
	\]
	
	\vspace{1cm}
	\begin{tikzpicture}
	\begin{axis}[xmin=-2,xmax=12, ymin=-1, ymax=1.2, axis lines=center, xlabel=y, ylabel=F(y)]
	\addplot[color=red, domain=-2:0]{0};
	\addplot[color=red, domain=0:1]{1/36};
	\addplot[color=red, domain=1:2]{3/36};
	\addplot[color=red, domain=2:3]{6/36};
	\addplot[color=red, domain=3:4]{10/36};
	\addplot[color=red, domain=4:5]{15/36};
	\addplot[color=red, domain=5:6]{21/36};
	\addplot[color=red, domain=6:7]{26/36};
	\addplot[color=red, domain=7:8]{30/36};
	\addplot[color=red, domain=8:9]{33/36};
	\addplot[color=red, domain=9:10]{35/36};
	\addplot[color=red, domain=10:11]{1};
	\end{axis}
	\end{tikzpicture}
\end{center}

\vspace{1cm}
c) Determine os valores de E(X) e V(X)

\begin{center}
	\vspace{0.25cm}
	$E(X) = \sum_{i}^{n} x_{i}*\mathbb{P}(X = x_{i})$
	
	\vspace{1cm}
	$E(X) = 1*\dfrac{1}{36}+2*\dfrac{3}{20}+3*\dfrac{5}{20} + 4*\dfrac{7}{20} + 5*\dfrac{9}{36} + 6*\dfrac{11}{36}$
	
	\vspace{0.25cm}
	$E(X) = \dfrac{1}{36} + \dfrac{6}{36} + \dfrac{15}{36} + \dfrac{28}{36} + \dfrac{45}{36} + \dfrac{66}{36} = \dfrac{161}{36}$
	
	\vspace{1cm}
	$E(X^{2}) = 1^{2}*\dfrac{1}{36}+2^{2}*\dfrac{3}{36}+3^{2}*\dfrac{5}{36} + 4^{2}*\dfrac{7}{36} + 5^{2}*\dfrac{9}{36} + 6^{2}*\dfrac{11}{36}$
	
	\vspace{0.25cm}
	$ E (X^{2})= \dfrac{1}{36} + \dfrac{12}{36} + \dfrac{45}{36} + \dfrac{112}{36} + \dfrac{225}{36} + \dfrac{396}{36} = \dfrac{791}{36}$
	
	\vspace{1cm}
	$V(X) = E(X^{2}) - E^{2}(X)$
	
	\vspace{1cm}
	$V(X) = \dfrac{791}{36}-\left(\dfrac{161}{36}\right)^{2} = \dfrac{28.476}{1.296}-\dfrac{25.921}{1.296} = \dfrac{2.555}{1.296}$
\end{center}

\begin{center}
	\vspace{1.5cm}
	$E(Y) = \sum_{i}^{n} y_{i}*\mathbb{P}(Y = y_{i})$
	
	\vspace{1cm}
	$E(Y) = 2*\dfrac{1}{36}+3*\dfrac{2}{20}+4*\dfrac{3}{20} + 5*\dfrac{4}{20} + 6*\dfrac{5}{36} + 7*\dfrac{6}{36} + 8*\dfrac{5}{36} + 9*\dfrac{4}{36} + 10*\dfrac{3}{36} + 11*\dfrac{2}{36} + 12*\dfrac{1}{36}$
	
	\vspace{0.25cm}
	$E(Y) = \dfrac{2}{36} + \dfrac{6}{36} + \dfrac{12}{36} + \dfrac{20}{36} + \dfrac{30}{36} + \dfrac{42}{36} + \dfrac{40}{36} + \dfrac{30}{36} + \dfrac{22}{36}  + \dfrac{12}{36} = \dfrac{251}{36}$
	
	\vspace{1cm}
	$E(Y^{2}) = 2^{2}*\dfrac{1}{36}+3^{2}*\dfrac{2}{36}+4^{2}*\dfrac{3}{36} + 5^{2}*\dfrac{4}{36} + 6^{2}*\dfrac{5}{36} + 7^{2}*\dfrac{6}{36} + 8^{2}*\dfrac{5}{36} + 9^{2}*\dfrac{4}{36} + 10^{2}*\dfrac{3}{36}+ 11^{2}*\dfrac{2}{36}+12^{2}*\dfrac{1}{36}$
	
	\vspace{0.25cm}
	$ E (X^{2})= \dfrac{4}{36} + \dfrac{18}{36} + \dfrac{48}{36} + \dfrac{100}{36} + \dfrac{180}{36} + \dfrac{295}{36} + \dfrac{320}{36} + \dfrac{324}{320} + \dfrac{242}{36} + \dfrac{144}{36}= \dfrac{1974}{36}$
	
	\vspace{1cm}
	$V(Y) = E(Y^{2}) - E^{2}(Y)$
	
	\vspace{1cm}
	$V(X) = \dfrac{1974}{36}-\left(\dfrac{251}{36}\right)^{2} = \dfrac{71.064}{1.296}-\dfrac{63.001}{1.296} = \dfrac{8.023}{1.296}$
\end{center}`

\vspace{1cm}
5) A distribuição de probabilidade de uma variável aleatória discreta é dada por:

\begin{table}[h]
\centering
\begin{tabular}{|c|c|c|c|}\hline
x & 1 & 2 & 3\\\hline
P(X=x) & 2a & a & 4a\\\hline
\end{tabular}
\end{table}

a) Determine o valor de a

\begin{center}
	\vspace{0.25cm}
	$\mathbb{P}(X=1) + \mathbb{P}(X=2)+ \mathbb{P}(X=3) = 1$
	\vspace{0.25cm}\\
	$2a+a+4a = 1$
	\vspace{0.25cm}\\
	$7a = 1$
	\vspace{0.25cm}\\
	$a= \dfrac{1}{7}$
\end{center}

\vspace{1cm}
b) Calcule as seguintes probabilidades: $\mathbb{P}(0\leq{x}\leq{3})$ e $\mathbb{P}(0<x<2)$

\vspace{0.25cm}
\begin{center}
	$\mathbb{P}(0\leq{x}\leq{3})= \mathbb{P}(X=1) + \mathbb{P}(X=2)+ \mathbb{P}(X=3)$
	\vspace{0.25cm}\\
	$\mathbb{P}(0\leq{x}\leq{3})=1$
	
	\vspace{1cm}
	$\mathbb{P}(0<x<2) = \mathbb{P}(X=1) = 2*\dfrac{1}{7} = \dfrac{2}{7}$
\end{center}

\vspace{1cm}
c) Encontre a função distribuição acumulada e faça sua representação gráfica

\begin{center}
	\vspace{0.25cm}
	$\mathbb{P}(X\leq{1}) = \mathbb{P}(X=1) = \dfrac{2}{7}$
	
	\vspace{0.5cm}
	$\mathbb{P}(X\leq2) = \mathbb{P}(X\leq{1}) + \mathbb{P}(X=2) = \dfrac{2}{7} + \dfrac{1}{7} = \dfrac{3}{7}$
	
	\vspace{0.5cm}
	$\mathbb{P}(X\leq{3}) = \mathbb{P}(X\leq2) + \mathbb{P}(X=3) = \dfrac{3}{7} + \dfrac{4}{7} = 1$
	
	\[
	F(x) =
	\begin{cases}
	0, 1 < x \\
	\frac{2}{7}, 1\leq{x} < 2 \\
	\frac{3}{7}, 2\leq{x} < 3\\
	1,x\geq{3}
	\end{cases}
	\]
	
	\vspace{1cm}
	\begin{tikzpicture}
	\begin{axis}[xmin=-2,xmax=5, ymin=-1, ymax=1.2, axis lines=center, xlabel=x, ylabel=F(x)]
	\addplot[color=red, domain=-2:0]{0};
	\addplot[color=red, domain=0:1]{2/7};
	\addplot[color=red, domain=1:2]{3/7};
	\addplot[color=red, domain=3:5]{1};
	\end{axis}
	\end{tikzpicture}
\end{center}

\vspace{1cm}
d) Determine os valores de E(X) e V(X)

\begin{center}
	\vspace{0.25cm}
	$E(X) = \sum_{i}^{n} x_{i}*\mathbb{P}(X = x_{i})$
	
	\vspace{1cm}
	$E(X) = 1*\dfrac{2}{7}+2*\dfrac{1}{7}+3*\dfrac{4}{7} = \dfrac{2}{7} + \dfrac{2}{7} + \dfrac{12}{7} = \dfrac{16}{7}$
	
	\vspace{0.5cm}
	$E(X^{2}) = 1^{2}*\dfrac{2}{7}+2^{2}*\dfrac{1}{7}+3^{2}*\dfrac{4}{7} = \dfrac{2}{7} + \dfrac{4}{7} + \dfrac{36}{7} = \dfrac{42}{7}$
	
	\vspace{1cm}
	$V(X) = E(x^{2}) - E^{2}(x)$
	
	\vspace{1cm}
	$V(X) = \dfrac{42}{7}-\left(\dfrac{16}{7}\right)^{2} = \dfrac{294}{49}-\dfrac{256}{49} = \dfrac{38}{49}$
\end{center}

e) Definindo-se Y=3X, determine E(Y) e V(Y)

\begin{center}
	\vspace{0.25cm}
	$E(Y) = E(3X) = 3*\dfrac{16}{7} = \dfrac{48}{7}$ 
	
	\vspace{1cm}
	$V(Y) = V(3X) = 3^{2}*V(X) = 9*\dfrac{38}{49} = \dfrac{2390}{49}$
\end{center}

\vspace{1cm}
6) De um lote com 5 bolas brancas, 3 verdes e 6 azuis serão retiradas 2 bolas. Encontre a distribuição de probabilidade da variável aleatória X=número de bolas brancas retiradas e de Y=número de bolas verdes retiradas, nos seguintes casos:

a) Retirada um a um com reposição

\[
S_{x} =
\begin{cases}
(B^{c},B^{c}) = 0 \\
(B^{c},B);(B,B^{c}) = 1\\
(B,B) = 2
\end{cases}
\]

\begin{center}
	\vspace{0.5cm}
	$\mathbb{P}(X=0) = \dfrac{9}{14}*\dfrac{9}{14}$ = $\dfrac{81}{196}$
	
	\vspace{0.5cm}
	$\mathbb{P}(X=1) = \dfrac{9}{14}*\dfrac{5}{14} + \dfrac{5}{14}*\dfrac{9}{14}$ = $\dfrac{45}{196} + \dfrac{45}{196}$ = $\dfrac{90}{196}$
	
	\vspace{0.5cm}
	$\mathbb{P}(X=2) = \dfrac{5}{14}*\dfrac{5}{14}$ = $\dfrac{25}{196}$
\end{center}

\vspace{1cm}
\[
S_{y} =
\begin{cases}
(V^{c},V^{c}) = 0 \\
(V^{c},V);(V,V^{c}) = 1\\
(V,V) = 2
\end{cases}
\]

\begin{center}
	\vspace{0.5cm}
	$\mathbb{P}(X=0) = \dfrac{11}{14}*\dfrac{11}{14}$ = $\dfrac{121}{196}$
	
	\vspace{0.5cm}
	$\mathbb{P}(X=1) = \dfrac{11}{14}*\dfrac{3}{14} + \dfrac{3}{14}*\dfrac{11}{14}$ = $\dfrac{33}{196} + \dfrac{3}{196}$ = $\dfrac{66}{196}$
	
	\vspace{0.5cm}
	$\mathbb{P}(X=2) = \dfrac{3}{14}*\dfrac{3}{14}$ = $\dfrac{9}{196}$
\end{center}

\vspace{1cm}
b) Retirada um a um sem reposição

\[
S_{x} =
\begin{cases}
(B^{c},B^{c}) = 0 \\
(B^{c},B);(B,B^{c}) = 1\\
(B,B) = 2
\end{cases}
\]

\begin{center}
	\vspace{0.5cm}
	$\mathbb{P}(X=0) = \dfrac{9}{14}*\dfrac{8}{13}$ = $\dfrac{72}{182}$
	
	\vspace{0.5cm}
	$\mathbb{P}(X=1) = \dfrac{9}{14}*\dfrac{5}{13} + \dfrac{5}{14}*\dfrac{9}{13}$ = $\dfrac{45}{182} + \dfrac{45}{182}$ = $\dfrac{90}{182}$
	
	\vspace{0.5cm}
	$\mathbb{P}(X=2) = \dfrac{5}{14}*\dfrac{4}{13}$ = $\dfrac{20}{182}$
\end{center}

\vspace{1cm}
\[
S_{y} =
\begin{cases}
(V^{c},V^{c}) = 0 \\
(V^{c},V);(V,V^{c}) = 1\\
(V,V) = 2
\end{cases}
\]

\begin{center}
	\vspace{0.5cm}
	$\mathbb{P}(X=0) = \dfrac{11}{14}*\dfrac{10}{13}$ = $\dfrac{110}{182}$
	
	\vspace{0.5cm}
	$\mathbb{P}(X=1) = \dfrac{11}{14}*\dfrac{3}{13} + \dfrac{3}{14}*\dfrac{11}{13}$ = $\dfrac{33}{182} + \dfrac{3}{182}$ = $\dfrac{66}{182}$
	
	\vspace{0.5cm}
	$\mathbb{P}(X=2) = \dfrac{3}{14}*\dfrac{2}{13}$ = $\dfrac{6}{182}$
\end{center}

\vspace{1cm}
Em cada caso:

c) Encontre a função distribuição acumulada e faça sua representação gráfica

\vspace{0.25cm}
COM REPOSIÇÃO
\begin{center}
	\vspace{0.25cm}
	$\mathbb{P}(X\leq{0}) = \mathbb{P}(X=0) = \dfrac{81}{196}$
	
	\vspace{0.5cm}
	$\mathbb{P}(X\leq1) = \mathbb{P}(X\leq{0}) + \mathbb{P}(X=1) = \dfrac{81}{196} + \dfrac{90}{196} = \dfrac{171}{196}$
	
	\vspace{0.5cm}
	$\mathbb{P}(X\leq{2}) = \mathbb{P}(X\leq1) + \mathbb{P}(X=2) = \dfrac{171}{196} + \dfrac{25}{196} = 1$
	
	\[
	F(y) =
	\begin{cases}
	0, 0 < x \\
	\frac{81}{196}, 0\leq{x} < 1 \\
	\frac{171}{196}, 1\leq{x} < 2\\
	1,x\geq{2}
	\end{cases}
	\]
	
	\vspace{1cm}
	\begin{tikzpicture}
	\begin{axis}[xmin=-2,xmax=4, ymin=-1, ymax=1.2, axis lines=center, xlabel=x, ylabel=F(x)]
	\addplot[color=red, domain=-2:0]{0};
	\addplot[color=red, domain=0:1]{81/196};
	\addplot[color=red, domain=1:2]{171/197};
	\addplot[color=red, domain=2:4]{1};
	\end{axis}
	\end{tikzpicture}
\end{center}

\vspace{1cm}

\begin{center}
	\vspace{0.25cm}
	$\mathbb{P}(Y\leq{0}) = \mathbb{P}(Y=0) = \dfrac{121}{196}$
	
	\vspace{0.5cm}
	$\mathbb{P}(Y\leq1) = \mathbb{P}(Y\leq{0}) + \mathbb{P}(Y=1) = \dfrac{121}{196} + \dfrac{66}{196} = \dfrac{187}{196}$
	
	\vspace{0.5cm}
	$\mathbb{P}(Y\leq{2}) = \mathbb{P}(Y\leq1) + \mathbb{P}(Y=2) = \dfrac{187}{196} + \dfrac{9}{196} = 1$
	
	\[
	F(y) =
	\begin{cases}
	0, 0 < y \\
	\frac{121}{196}, 0\leq{y} < 1 \\
	\frac{187}{196}, 1\leq{y} < 2\\
	1,y\geq{2}
	\end{cases}
	\]
	
	\vspace{1cm}
	\begin{tikzpicture}
	\begin{axis}[xmin=-2,xmax=4, ymin=-1, ymax=1.2, axis lines=center, xlabel=y, ylabel=F(y)]
	\addplot[color=red, domain=-2:0]{0};
	\addplot[color=red, domain=0:1]{121/196};
	\addplot[color=red, domain=1:2]{187/197};
	\addplot[color=red, domain=2:4]{1};
	\end{axis}
	\end{tikzpicture}
\end{center}

\vspace{1cm}
SEM REPOSIÇÃO
\begin{center}
	\vspace{0.25cm}
	$\mathbb{P}(X\leq{0}) = \mathbb{P}(X=0) = \dfrac{72}{182}$
	
	\vspace{0.5cm}
	$\mathbb{P}(X\leq1) = \mathbb{P}(X\leq{0}) + \mathbb{P}(X=1) = \dfrac{72}{182} + \dfrac{90}{182} = \dfrac{162}{182}$
	
	\vspace{0.5cm}
	$\mathbb{P}(X\leq{2}) = \mathbb{P}(X\leq1) + \mathbb{P}(X=2) = \dfrac{162}{182} + \dfrac{20}{182} = 1$
	
	\[
	F(x) =
	\begin{cases}
	0, 0 < x \\
	\frac{72}{182}, 0\leq{x} < 1 \\
	\frac{162}{182}, 1\leq{x} < 2\\
	1,x\geq{2}
	\end{cases}
	\]
	
	\vspace{1cm}
	\begin{tikzpicture}
	\begin{axis}[xmin=-2,xmax=4, ymin=-1, ymax=1.2, axis lines=center, xlabel=x, ylabel=F(x)]
	\addplot[color=red, domain=-2:0]{0};
	\addplot[color=red, domain=0:1]{72/182};
	\addplot[color=red, domain=1:2]{162/182};
	\addplot[color=red, domain=2:4]{1};
	\end{axis}
	\end{tikzpicture}
\end{center}

\vspace{1cm}

\begin{center}
	\vspace{0.25cm}
	$\mathbb{P}(Y\leq{0}) = \mathbb{P}(Y=0) = \dfrac{110}{182}$
	
	\vspace{0.5cm}
	$\mathbb{P}(Y\leq1) = \mathbb{P}(Y\leq{0}) + \mathbb{P}(Y=1) = \dfrac{110}{182} + \dfrac{66}{182} = \dfrac{176}{182}$
	
	\vspace{0.5cm}
	$\mathbb{P}(Y\leq{2}) = \mathbb{P}(Y\leq1) + \mathbb{P}(Y=2) = \dfrac{176}{182} + \dfrac{6}{182} = 1$
	
	\[
	F(y) =
	\begin{cases}
	0, 0 < y \\
	\dfrac{110}{182}, 0\leq{y} < 1 \\
	\dfrac{176}{182}, 1\leq{y} < 2\\
	1,y\geq{2}
	\end{cases}
	\]
	
	\vspace{1cm}
	\begin{tikzpicture}
	\begin{axis}[xmin=-2,xmax=4, ymin=-1, ymax=1.2, axis lines=center, xlabel=y, ylabel=F(y)]
	\addplot[color=red, domain=-2:0]{0};
	\addplot[color=red, domain=0:1]{110/182};
	\addplot[color=red, domain=1:2]{176/182};
	\addplot[color=red, domain=2:4]{1};
	\end{axis}
	\end{tikzpicture}
\end{center}

\vspace{1cm}
d) Determine os valores de E(X) e V(X)

\vspace{0.25cm}
COM REPOSIÇÃO
\begin{center}
	\vspace{0.25cm}
	$E(X) = \sum_{i}^{n} x_{i}*\mathbb{P}(X = x_{i})$
	
	\vspace{1cm}
	$E(X) = 0*\dfrac{81}{196}+1*\dfrac{90}{196}+2*\dfrac{25}{196}= 0 + \dfrac{90}{196} + \dfrac{50}{196} = \dfrac{140}{196} = \dfrac{70}{98}$
	
	\vspace{0.5cm}
	$E(X^{2}) = 0^{2}*\dfrac{81}{196}+1^{2}*\dfrac{90}{196}+2^{2}*\dfrac{25}{196} = 0 + \dfrac{90}{196} + \dfrac{100}{196} = \dfrac{190}{196} = \dfrac{95}{98}$
	
	\vspace{1cm}
	$V(X) = E(X^{2}) - E^{2}(X)$
	
	\vspace{1cm}
	$V(X) = \dfrac{95}{98}-(\dfrac{70}{98})^{2} = \dfrac{9.310}{9.604}-\dfrac{6.860}{9.604}=\dfrac{2.450}{9.604}$
	
	\vspace{1.5cm}
	$E(Y) = \sum_{i}^{n} y_{i}*\mathbb{P}(Y = y_{i})$
	
	\vspace{1cm}
	$E(Y) = 0*\dfrac{121}{196}+1*\dfrac{66}{196}+2*\dfrac{9}{196}= 0 + \dfrac{66}{196} + \dfrac{18}{196} = \dfrac{84}{196} = \dfrac{42}{98}$
	
	\vspace{0.5cm}
	$E(Y^{2}) = 0^{2}*\dfrac{121}{196}+1^{2}*\dfrac{66}{196}+2^{2}*\dfrac{9}{196} = 0 + \dfrac{66}{196} + \dfrac{36}{196} = \dfrac{102}{196} = \dfrac{51}{98}$
	
	\vspace{1cm}
	$V(Y) = E(Y^{2}) - E^{2}(Y)$
	
	\vspace{1cm}
	$V(Y) = \dfrac{51}{98}-(\dfrac{42}{98})^{2} = \dfrac{4.998}{9.604}-\dfrac{1.764}{9.604} = \dfrac{3.234}{9.604}$
\end{center}

\vspace{1cm}
SEM REPOSIÇÃO
\begin{center}
	\vspace{0.25cm}
	$E(X) = \sum_{i}^{n} x_{i}*\mathbb{P}(X = x_{i})$
	
	\vspace{1cm}
	$E(X) = 0*\dfrac{72}{182}+1*\dfrac{90}{182}+2*\dfrac{20}{182} = 0 + \dfrac{90}{182} + \dfrac{40}{182} = \dfrac{130}{182} = \dfrac{65}{91}$
	
	\vspace{0.5cm}
	$E(X^{2}) = 0^{2}*\dfrac{72}{182}+1^{2}*\dfrac{90}{182}+2^{2}*\dfrac{20}{182} = 0 + \dfrac{90}{182} + \dfrac{80}{182} = \dfrac{170}{182} = \dfrac{85}{91}$
	
	\vspace{1cm}
	$V(X) = E(X^{2}) - E^{2}(X)$
	
	\vspace{1cm}
	$V(X) = \dfrac{85}{91}-(\dfrac{65}{91})^{2} = \dfrac{7.735}{8.281}-\dfrac{4.225}{8.281} = \dfrac{3.510}{8.281}$

	\vspace{1.5cm}
	$E(Y) = \sum_{i}^{n} y_{i}*\mathbb{P}(Y = y_{i})$

	\vspace{1cm}
	$E(Y) = 0*\dfrac{110}{182}+1*\dfrac{66}{182}+2*\dfrac{6}{196}= 0 + \dfrac{66}{192} + \dfrac{12}{182} = \dfrac{74}{182} = \dfrac{37}{91}$

	\vspace{0.5cm}
	$E(Y^{2}) = 0^{2}*\dfrac{110}{182}+1^{2}*\dfrac{66}{182}+2^{2}*\dfrac{6}{182} = 0 + \dfrac{66}{182} + \dfrac{24}{182} = \dfrac{90}{182} = \dfrac{45}{91}$

	\vspace{1cm}
	$V(Y) = E(Y^{2}) - E^{2}(Y)$

	\vspace{1cm}
	$V(Y) = \dfrac{45}{91}-(\dfrac{37}{91})^{2} = \dfrac{4.085}{8.281}-\dfrac{1.369}{8.281} = \dfrac{2.716}{8.281}$
	\end{center}

\vspace{1cm}
7) Uma moeda não viciada é lançada 3 vezes. Encontre a distribuição de probabilidade da variável aleatória X= número de coroas que apareceram.

a) Encontre a distribuição de probabilidade

\begin{center}
	\[
	S_{x} =
	\begin{cases}
	(k,k,k) = 0 \\
	(k,k,c);(k,c,k);(c,k,k) = 1\\
	(k,c,c);(c,k,c);(c,c,k) = 2\\
	(k,k,k) = 3
	\end{cases}
	\]
	
	\vspace{0.5cm}
	$\mathbb{P}(X=0) = \dfrac{1}{2}*\dfrac{1}{2}*\dfrac{1}{2} = \dfrac{1}{8}$
	
	\vspace{0.5cm}
	$\mathbb{P}(X=1) = \dfrac{1}{2}*\dfrac{1}{2}*\dfrac{1}{2}*3 = \dfrac{1}{8}*3 = \dfrac{3}{8}$
	
	\vspace{0.5cm}
	$\mathbb{P}(X=2) = \dfrac{1}{2}*\dfrac{1}{2}*\dfrac{1}{2}*3 = \dfrac{1}{8}*3 = \dfrac{3}{8}$
	
	\vspace{0.5cm}
	$\mathbb{P}(X=3) = \dfrac{1}{2}*\dfrac{1}{2}*\dfrac{1}{2} = \dfrac{1}{8}$
\end{center}

\vspace{1cm}
b)  Encontre a função distribuição acumulada e faça sua representação gráfica
\begin{center}
	\vspace{0.25cm}
	$\mathbb{P}(X\leq{0}) = \mathbb{P}(X=0) = \dfrac{1}{8}$
	
	\vspace{0.5cm}
	$\mathbb{P}(X\leq1) = \mathbb{P}(X\leq{0}) + \mathbb{P}(X=1) = \dfrac{1}{8} + \dfrac{3}{8} = \dfrac{4}{8}$
	
	\vspace{0.5cm}
	$\mathbb{P}(X\leq{2}) = \mathbb{P}(X\leq1) + \mathbb{P}(X=2) = \dfrac{4}{8} + \dfrac{3}{8} = \dfrac{7}{8}$
	
	\vspace{0.5cm}
	$\mathbb{P}(X\leq{3}) = \mathbb{P}(X\leq{2}) + \mathbb{P}(X=3) = \dfrac{7}{8} + \dfrac{1}{8} = \dfrac{8}{8} = 1$
	
	\[
	F(x) =
	\begin{cases}
	0, 0 < x \\
	\frac{1}{8}, 0\leq{x} < 1 \\
	\frac{4}{8}, 1\leq{x} < 2\\
	\frac{7}{8}, 2\leq{x} < 3\\
	1,x\geq{3}
	\end{cases}
	\]
	
	\vspace{1cm}
	\begin{tikzpicture}
	\begin{axis}[xmin=-2,xmax=5, ymin=-1, ymax=1.2, axis lines=center, xlabel=x, ylabel=F(x)]
	\addplot[color=red, domain=-2:0]{0};
	\addplot[color=red, domain=0:1]{1/8};
	\addplot[color=red, domain=1:2]{4/8};
	\addplot[color=red, domain=2:3]{7/8};
	\addplot[color=red, domain=3:5]{1};
	\end{axis}
	\end{tikzpicture}
\end{center}

\vspace{1cm}
c) Determine os valores de E(X) e V(X)

\begin{center}
	\vspace{0.25cm}
	$E(X) = \sum_{i}^{n} x_{i}*\mathbb{P}(X = x_{i})$
	
	\vspace{1cm}
	$E(X) = 0*\dfrac{1}{8}+1*\dfrac{3}{8}+2*\dfrac{3}{8} + 3*\dfrac{1}{8}= 0 + \frac{3}{8} + \frac{6}{8} + \frac{3}{8} = \dfrac{12}{8} = \dfrac{3}{2}$
	
	\vspace{0.5cm}
	$E(X^{2}) = 0^{2}*\dfrac{1}{8}+1^{2}*\dfrac{3}{8}+2^{2}*\dfrac{3}{8} + 3^{2}*\dfrac{1}{8} = 0 + \dfrac{3}{8} + \dfrac{12}{8} + \dfrac{9}{8} = \dfrac{24}{8} = \dfrac{6}{2}$
	
	\vspace{1cm}
	$V(X) = E(X^{2}) - E^{2}(X)$
	
	\vspace{1cm}
	$V(X) = \dfrac{6}{2}-(\dfrac{3}{2})^{2} = \dfrac{12}{4}-\dfrac{9}{4}= \dfrac{3}{4}$
\end{center}

\vspace{1cm}
8) Uma variável aleatória discreta X pode assumir os valores 5; 10 e 15. Sabendo que F(5)=0,35, F(10)= 0,70 e F(15)=1, faça a representação gráfica de F(x) e encontre os valores de E(X) e V(X).

\begin{center}
	\vspace{0.25cm}
	$F(5) = \mathbb{P}(X\geq5) = \mathbb{P}(X=5)=0.35$
	
	\vspace{1cm}
	$F(10) = \mathbb{P}(X\geq10) = \mathbb{P}(X\geq5) + \mathbb{P}(X = 10) \Leftrightarrow \mathbb{P}(X = 10) = \mathbb{P}(X\geq10) - \mathbb{P}(X\geq5)$
	
	\vspace{0.25cm}
	$ \mathbb{P}(X = 10) = 0.70 - 0.35 = 0.35$
	
	\vspace{1cm}
	$F(15) = \mathbb{P}(X\geq15) = \mathbb{P}(X\geq10) + \mathbb{P}(X = 15) \Leftrightarrow \mathbb{P}(X = 15) = \mathbb{P}(X\geq15) - \mathbb{P}(X\geq10)$
	
	\vspace{0.25cm}
	$ \mathbb{P}(X = 10) = 1 - 0.7 = 0.3$
	
	\vspace{1.5cm}
	$E(X) = \sum_{i}^{n} x_{i}*\mathbb{P}(X = x_{i})$
	
	\vspace{1cm}
	$E(X) = 5*0.35+10*0.35+15*0.3= 1.75 + 3.5 + 4.5 = 9.75$
	
	\vspace{0.5cm}
	$E(X^{2}) = 5^{2}*0.35+10^{2}*0.35+15^{2}*0.3 = 8.75 + 35 + 67.5 = 111.25$
	
	\vspace{1cm}
	$V(X) = E(X^{2}) - E^{2}(X)$
	
	\vspace{1cm}
	$V(X) = 111.25-(9.75)^{2} = 111,25-95.0625 = 16.1875$
\end{center}

\vspace{1cm}
9) Aos valores da v.a. X da questão 8 foi somado o número 4 definindo-se Y=4+X. Determine E(Y) e V(Y).

\begin{center}
	\vspace{0.25cm}
	$E(Y) = E(4 + X) = 4 + E(X) = 4 + 9,75 = 13,75$ 
	
	\vspace{1cm}
	$V(Y) = V(4 + X) = V(X) = 111,75$
\end{center}

\vspace{1cm}
10) Para produzir um determinado componente eletrônico se gasta R\$50,00. Este componente é vendido por R\$100,00. Um lote de 25 componentes é posto a venda. Sabe-se que no lote tem apenas 2 componentes com defeito. O comprador vai inspecionar 2 componentes e comprará o lote se encontrar no máximo um componente defeituoso. Qual o lucro esperado do produtor?
\vspace{0.5cm}\\
X = Selecionar componentes perfeitos
\begin{center}	
	\[
	S_{x} =
	\begin{cases}
	(D,D) = 0 \\
	(D,P);(P,D) = 1\\
	(P,P) = 2
	\end{cases}
	\]
\end{center}
\vspace{1cm}
Y = Vender o lote
\begin{center}
	\[
	S_{x} =
	\begin{cases}
	(P,P); (P,D); (D,P) = s \\
	(D,D) = n
	\end{cases}
	\]
\end{center}

\vspace{0.25cm}
COM REPOSIÇÃO

\begin{center}	
	\vspace{0.5cm}
	$\mathbb{P}$(X = 0) = $\dfrac{2}{25}*\dfrac{2}{25}$ = $\dfrac{4}{625}$
	
	\vspace{0.5cm}
	$\mathbb{P}$(X = 1) = $\dfrac{2}{25}*\dfrac{23}{25} + \dfrac{23}{25}*\dfrac{2}{25}$ = $\dfrac{46}{625} + \dfrac{46}{625}$ = $\dfrac{92}{625}$
	
	\vspace{0.5cm}
	$\mathbb{P}$(X = 2) = $\dfrac{23}{25}*\dfrac{23}{25}$ = $\dfrac{529}{625}$

	\vspace{1cm}
	$\mathbb{P}$(Y = s) = $\dfrac{92}{625} + \dfrac{529}{625}$ = $\dfrac{621}{625}$

	\vspace{0.5cm}
	$\mathbb{P}$(Y = n) = $\dfrac{4}{625}$

	\vspace{1cm}
	Lucro = $ 25\left[(100 - 50) * \dfrac{621}{625} + (-50) * \dfrac{4}{625}\right] = 1.242 - 8 = 1.234$
\end{center}

\vspace{1.5cm}
SEM REPOSIÇÃO

\begin{center}	
	\vspace{0.5cm}
	$\mathbb{P}(X=0) = \dfrac{2}{25}*\dfrac{1}{24}$ = $\dfrac{2}{600}$
	
	\vspace{0.5cm}
	$\mathbb{P}(X=1) = \dfrac{2}{25}*\dfrac{23}{24} + \dfrac{23}{25}*\dfrac{2}{23}$ = $\dfrac{46}{600} + \dfrac{46}{600}$ = $\dfrac{92}{600}$
	
	\vspace{0.5cm}
	$\mathbb{P}(X=2) = \dfrac{23}{25}*\dfrac{22}{24}$ = $\dfrac{506}{600}$
	
	\vspace{1cm}
	$\mathbb{P}$(Y = s) = $\dfrac{92}{600} + \dfrac{506}{600}$ = $\dfrac{596}{600}$
	
	\vspace{0.5cm}
	$\mathbb{P}$(Y = n) = $\dfrac{4}{600}$
	
	\vspace{1cm}
	Lucro = $ 25\left[(100 - 50) * \dfrac{596}{600} + (-50) * \dfrac{4}{600}\right] = 1.241,66 - 8,33 = 1.233,33$
\end{center}

\vspace{1cm}
11) A função distribuição acumulada, F(x), de uma v.a. discreta X é dada por:

F(x) = 0, x < -2

F(x)= 0,25, se $-2\leq{x}< 1$

F(x)= 0,40, se  $1\leq{x}< 3$

F(x)= 0,70, se $3\leq{x}< 5$

F(x)= 1 se $x\geq{5}$

Encontre: $\mathbb{P}(x=3)$, $\mathbb{P}(x=4)$, F(0) e F(4) 

\begin{center}
	\vspace{0.25cm}
	$\mathbb{P}(X = 3) = F(3) - F(1) = 0.7 - 0.4 = 0.3$
	
	\vspace{0.5cm}
	$\mathbb{P}(X = 4) = 0$
	
	\vspace{0.5cm}
	$F(0) = \mathbb{P}(-2\leq{x}< 1) = 0.25$
	
	\vspace{0.5cm}
	$F(4) = \mathbb{P}(3\leq{x}< 5) = 0.7$
\end{center}

\end{document}