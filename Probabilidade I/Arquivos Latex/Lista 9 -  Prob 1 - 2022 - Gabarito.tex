\documentclass[12pt,a4paper,draft]{article}
\usepackage[utf8]{inputenc}
\usepackage[T1]{fontenc}
\usepackage{amsmath}
\usepackage{amsfonts}
\usepackage{amssymb}
\usepackage{graphicx}
\usepackage[width=0.00cm, left=3.00cm, right=2.00cm, top=3.00cm, bottom=2.00cm]{geometry}
\title{Lista 9}
\date{}
\begin{document}
	\maketitle
	\begin{center}
		\textbf{Curso de Ciências atuariais}\\
		\textbf{Disciplina Probabilidade 1- Professora Cristina}\\
		\textbf{29/08/2022 - Exercícios distribuição geométrica e hipergeométrica}
	\end{center}
	1) Considere o experimento em que uma moeda viciada é lançada sucessivas vezes, até que ocorra a primeira cara. Seja  a variável aleatória que conta o número de coroas obtidos no experimento (ou seja, a quantidade de lançamentos anteriores a obtenção da primeira cara). Sabendo que a probabilidade de cara é de $\frac{3}{5}$, qual é a probabilidade de $\mathbb{P}$(2 $\leq$ X < 4)
	\vspace{0.5cm}\\
	Y = Obter cara no lançamento de uma moeda
	\begin{center}
		\vspace{0.5cm}
		$\mathbb{P}$(Y = k) = $\dfrac{3}{5}$
	\end{center}
	\vspace{1cm}
	X = Obter a primeira cara no lançamento de uma  moeda
	\begin{center}
		\vspace{0.5cm}
		$\mathbb{P}$(2 $\leq$ X < 4) = $\mathbb{P}$(X = 2) + $\mathbb{P}$(X = 3)
		\vspace{0.5cm}\\
		$\mathbb{P}$(2 $\leq$ X < 4) = $\dfrac{2}{5} * \dfrac{3}{5} + \left(\dfrac{2}{5}\right)^2 * \dfrac{3}{5}$ = $\dfrac{6}{25} + \dfrac{12}{125}$ = $\dfrac{42}{125}$
	\end{center}
	\vspace{1cm}
	2) Um pesquisador está realizando experimentos químicos independentes e sabe que a probabilidade de que cada experimento apresente uma reação positiva é 10\% . Qual é a probabilidade de que menos de 5 reações negativas ocorram antes da primeira positiva?
	\vspace{0.5cm}\\
	Y = Obter uma reação química positiva
	\begin{center}
		\vspace{0.5cm}
		$\mathbb{P}$(Y = s) = 10\% = 0,1
	\end{center}
	\vspace{1cm}
	X = Obter a primeira reação química positiva
	\begin{center}
		\vspace{0.5cm}
		$\mathbb{P}$(X < 5) = $\mathbb{P}$(X = 1) + $\mathbb{P}$(X = 2) + $\mathbb{P}$(X = 3) + $\mathbb{P}$(X = 4)
		\vspace{0.5cm}\\
		$\mathbb{P}$(X < 5) = $0,1 + 0,9 * 0,1 + \left(0,9\right)^2 * 0,1 + \left(0,9\right)^3 * 0,1$ 
		\vspace{0.5cm}\\
		$\mathbb{P}$(X < 5) = 0,01 + 0,09 + 0,081 + 0,0729 = 0,2539
	\end{center}
\vspace{1cm}
	3) Você está procurando emprego e está enviando seu CV (Currículo Vitae).	Apenas 25\% dos CVs enviados resultam numa entrevista. Calcule a probabilidade de que a primeira entrevista ocorrerá no envio do 10 o CV.
	\vspace{0.5cm}\\
	Y = Ser chamado para uma entrevista após enviar um currículo
	\begin{center}
		\vspace{0.5cm}
		$\mathbb{P}$(Y = s) = 25\% = 0,25
	\end{center}
	\vspace{1cm}
	X = Ser chamado para a primeira entrevista após enviar currículo
	\begin{center}
		\vspace{0.5cm}
		$\mathbb{P}$(X = 10) = $\left(0,75\right)^9 * 0,25 = 0,07508 * 0,25 = 0,01877$ 
	\end{center}
	\vspace{1cm}
	4) Um estado tem uma loteria em que seis números são selecionados
	aleatoriamente de 40, sem reposição. Um jogador escolhe seis números antes do sorteio acontecer.\\
	a) Qual é a probabilidade de que os seis números escolhidos pelo jogador coincidam com todos os seis números sorteados?
	\vspace{0.5cm}
	X = Acertar os número da loteria
	\begin{center}
		\vspace{0.5cm}
		$\mathbb{P}$(X = 6) = $\dfrac{\dbinom{6}{6} * \dbinom{34}{0}}{\dbinom{40}{6}} = \dfrac{\dfrac{6!}{6!0!} * \dfrac{34!}{34!0!}}{\dfrac{40!}{34! 6!}} = \dfrac{34!6!}{40!} = \dfrac{720}{2.763.633.600} = 0,0000002605$
	\end{center}
	\vspace{1cm}
	b) Qual é a probabilidade de que cinco dos seis números escolhidos pelo jogador apareçam entre os números sorteados?
	\begin{center}
		\vspace{0.5cm}
		$\mathbb{P}$(X = 5) = $\dfrac{\dbinom{6}{5} * \dbinom{34}{1}}{\dbinom{40}{6}} = \dfrac{\dfrac{6!}{5!1!} * \dfrac{34!}{33!1!}}{\dfrac{40!}{34! 6!}} = \dfrac{6 * 34 * 34!6!}{40!} = \dfrac{146.880}{2.763.633.600} = 0,0000531474$
	\end{center}
	\vspace{1cm}
	c) Qual é a probabilidade de que quatro dos seis números escolhidos pelo jogador apareçam entre os números sorteados?
	\begin{center}
		\vspace{0.5cm}
		$\mathbb{P}$(X = 4) = $\dfrac{\dbinom{6}{4} * \dbinom{34}{2}}{\dbinom{40}{6}} = \dfrac{\dfrac{6!}{4!2!} * \dfrac{34!}{32!2!}}{\dfrac{40!}{34! 6!}} = \dfrac{15 * 561 * 34!6!}{40!} = \dfrac{6.058.800}{2.763.633.600}$\\ 
		\vspace{0.5cm}
		$\mathbb{P}$(X = 4) = 0,0021923311
	\end{center}
	\vspace{1cm}
	d) Qual é a probabilidade de que no máximo cinco dos seis números escolhidos pelo jogador apareçam entre os números sorteados?
	\begin{center}
		\vspace{0.5cm}
		$\mathbb{P}$(X $\leq$ 5) = 1 - $\mathbb{P}$(X = 6)
		\vspace{0.5cm}\\
		$\mathbb{P}$(X $\leq$ 5) = 1 - 0,0000002605 = 0,9999997395
\end{center}
	\vspace{1cm}
	5) Cartões de circuito integrado são verificados em um teste funcional. Um lote contém 140 cartões e 20 são selecionados sem reposição para o teste funcional.\\
	a) Se 20 cartões forem defeituosos, qual será a probabilidade de que no mínimo um cartão defeituoso esteja na amostra?
	\vspace{0.5cm}\\
	X = Selecionar um cartão defeituoso
	\begin{center}
		\vspace{0.5cm}
		$\mathbb{P}$(X $\geq$ 1) = 1 - $\mathbb{P}$(X = 0)
		\vspace{0.5cm}\\
		$\mathbb{P}$(X $\geq$ 1) = 1 - $\dfrac{\dbinom{20}{0} * \dbinom{120}{20}}{\dbinom{140}{20}} = 1 - \dfrac{\dfrac{20!}{20!0!} * \dfrac{120!}{100!20!}}{\dfrac{140!}{120!20!}} = 1 - \dfrac{120!120!20!}{100!20!140!}$
		\vspace{0.5cm}\\
		$\mathbb{P}$(X $\geq$ 1) = 1 - 0,0356 = 0,9644
	\end{center}
	\vspace{1cm}
	b) Se 5 cartões forem defeituosos, qual será a probabilidade de que no mínimo um cartão defeituoso apareça na amostra?
	\begin{center}
		\vspace{0.5cm}
		$\mathbb{P}$(X $\geq$ 1) = 1 - $\mathbb{P}$(X = 0)
		\vspace{0.5cm}\\
		$\mathbb{P}$(X $\geq$ 1) = 1 - $\dfrac{\dbinom{5}{0} * \dbinom{135}{20}}{\dbinom{140}{20}} = 1 - \dfrac{\dfrac{5!}{5!0!} * \dfrac{135!}{115!20!}}{\dfrac{140!}{120!20!}} = 1 - \dfrac{135!120!20!}{115!20!140!}$
		\vspace{0.5cm}\\
		$\mathbb{P}$(X $\geq$ 1) = 1 - 0,4571 = 0,5429
	\end{center}
	\vspace{1cm}
	Um lote contém 100 partes de um fornecedor brasileiro e 200 partes de um fornecedor chinês. Se quatro partes são selecionadas aleatoriamente, sem	reposição, qual é a probabilidade de que sejam todas elas de um fornecedor	brasileiro? Qual é a probabilidade de que duas ou mais partes na amostra sejam de um fornecedor brasileiro? Qual a probabilidade de que pelo menos uma parte seja de um fornecedor brasileiro?
	\vspace{0.5cm}\\
	X = Selecionar partes de um fornecedor brasileiro
	\begin{center}
		$\mathbb{P}$(X = 4) = $\dfrac{\dbinom{100}{4} * \dbinom{200}{0}}{\dbinom{300}{4}} = \dfrac{\dfrac{100!}{4!96!} * \dfrac{200!}{0!200!}}{\dfrac{300!}{4!296!}} = \dfrac{100!4!296!}{4!96!300!}$
		\vspace{0.5cm}\\
		$\mathbb{P}$(X = 4) = 0,0118
		\vspace{1.5cm}\\
		$\mathbb{P}$(X $\geq$ 2) = 1 - $\mathbb{P}$(X < 2) = 1 - [$\mathbb{P}$(X = 0) + $\mathbb{P}$(X = 1)]
		\vspace{0.5cm}\\
		$\mathbb{P}$(X $\geq$ 2) = 1 - $\left[\dfrac{\dbinom{100}{0} * \dbinom{200}{4}}{\dbinom{300}{4}} + \dfrac{\dbinom{100}{1} * \dbinom{200}{3}}{\dbinom{300}{4}}\right]$ 
		\vspace{0.5cm}\\
		$\mathbb{P}$(X $\geq$ 2) = 1 - $\left[\dfrac{\dfrac{100!}{0!100!} * \dfrac{200!}{4!196!}}{\dfrac{300!}{4!296!}} + \dfrac{\dfrac{100!}{1!99!} * \dfrac{200!}{3!197!}}{\dfrac{300!}{4!296!}}\right]$
		\vspace{0.5cm}\\
		$\mathbb{P}$(X $\geq$ 2) = 1 - $\left[\dfrac{200!4!296!}{4!196!300!} + \dfrac{100 * 200!4!296!}{3!197!300!}\right]$
		\vspace{0.5cm}\\
		$\mathbb{P}$(X $\geq$ 1) = 1 - [0,1955 + 0,3970] = 1 - 0,5925 = 0,4075
		\vspace{1.5cm}\\
		$\mathbb{P}$(X $\geq$ 1) = 1 - $\mathbb{P}$(X = 0)
		\vspace{0.5cm}\\
		$\mathbb{P}$(X $\geq$ 1) = 1 - $\dfrac{\dbinom{100}{0} * \dbinom{200}{4}}{\dbinom{300}{4}} = 1 - \dfrac{\dfrac{100!}{0!100!} * \dfrac{200!}{4!196!}}{\dfrac{300!}{4!296!}} = 1 - \dfrac{200!4!296!}{4!196!300!}$
		\vspace{0.5cm}\\
		$\mathbb{P}$(X $\geq$ 1) = 1 - 0,1955 = 0,8045
	\end{center}
\end{document}
