\documentclass[12pt,a4paper]{article}
\usepackage[utf8]{inputenc}
\usepackage[T1]{fontenc}
\usepackage{amsmath}
\usepackage{amsfonts}
\usepackage{amssymb}
\usepackage{graphicx}
\usepackage[width=0.00cm, left=3.00cm, right=2.00cm, top=3.00cm, bottom=2.00cm]{geometry}
\title{Prova 3}
\usepackage{multirow}
\usepackage{booktabs}
\usepackage{tikz}
\date{}
\begin{document}
	\maketitle
	\begin{center}
		\textbf{Curso de Ciências Atuariais}\\
		\textbf{Disciplina Probabilidade 1- Professora Cristina}\\
		\textbf{26/09/2022 - Terceiro exercício de probabilidade}
	\end{center}
	1) Uma moeda honesta é lançada 3 vezes. Seja X o número de caras nos 2 primeiros lançamentos e seja Y o número de caras nos 2 últimos lançamentos.\\
	a) (0,5 pontos) Liste todos os elementos do espaço amostral deste experimento, especificando os valores de X e Y.\\
	\vspace{0.5cm}\\
	\[
	S_{X_{1}} =
	\begin{cases}
	(c, c, \_) = 0\\
	(c, k, \_); (k, c, \_) = 1\\
	(k, k, \_) = 2\\
	\end{cases}
	\]
	\vspace{0.5cm}\\
	\[
	S_{Y_{1}} =
	\begin{cases}
	(\_, c, c) = 0\\
	(\_, c, k); (\_, k, c) = 1\\
	(\_, k, k) = 2\\
	\end{cases}
	\]
	\vspace{0.5cm}\\
	\begin{center}
		\begin{tabular}{cccc}\midrule
			& X & Y & $\mathbb{P}$()\\ \midrule
			(c, c, c) & 0 & 0 & $\left(\dfrac{1}{2}\right)^3 = \dfrac{1}{8}$\\ \midrule
			(c, c, k) & 0 & 1 & $\left(\dfrac{1}{2}\right)^2 * \dfrac{1}{2} = \dfrac{1}{8}$\\ \midrule
			(c, k, c) & 1 & 1 & $\dfrac{1}{2} * \dfrac{1}{2} * \dfrac{1}{2}= \dfrac{1}{8}$\\ \midrule
			(k, c, c) & 1 & 0 & $\dfrac{1}{2} * \left(\dfrac{1}{2}\right)^2 = \dfrac{1}{8}$\\ \midrule
			(c, k, k) & 1 & 2 & $\dfrac{1}{2} * \left(\dfrac{1}{2}\right)^2 = \dfrac{1}{8}$\\ \midrule
			(k, c, k) & 1 & 1 & $\dfrac{1}{2} * \dfrac{1}{2} * \dfrac{1}{2}= \dfrac{1}{8}$\\ \midrule
			(k, k, c) & 2 & 1 & $\left(\dfrac{1}{2}\right)^2 * \dfrac{1}{2} = \dfrac{1}{8}$\\ \midrule
			(k, k, k) & 2 & 2 & $\left(\dfrac{1}{2}\right)^3 = \dfrac{1}{8}$\\ \midrule
		\end{tabular}
	\end{center}
	\vspace{1cm}
	b) (1 ponto) Construa a função de distribuição de probabilidade conjunta de (X,Y).\\
	\vspace{0.5cm}\\
	\begin{center}
		\begin{tabular}{ccccc}\midrule
			\multirow{2}{*}{Y} & \multicolumn{3}{c}{X} & \multirow{2}{*}{$\mathbb{P}$(Y = y)}\\ \cmidrule{2-4}
			& 0 & 1 & 2 & \\ \midrule
			0 & $\dfrac{1}{8}$ & $\dfrac{1}{8}$ & 0 & $\dfrac{2}{8}$\\ \midrule
			1 & $\dfrac{1}{8}$ & $\dfrac{2}{8}$ & $\dfrac{1}{8}$ & $\dfrac{4}{8}$\\ \midrule
			2 & 0 & $\dfrac{1}{8}$ & $\dfrac{1}{8}$ & $\dfrac{2}{8}$\\ \midrule
			$\mathbb{P}$(X = x) & $\dfrac{2}{8}$ & $\dfrac{4}{8}$ & $\dfrac{2}{8}$ & 1\\ \midrule
		\end{tabular}
	\end{center}
	\vspace{1cm}
	c) (1 ponto) Calcule E(X)\\
	\vspace{0.25cm}\\
	\begin{center}
		E(X) = $0 * \dfrac{2}{8} + 1 * \dfrac{4}{8} + 2 * \dfrac{2}{8} = 0 + \dfrac{4}{8} + \dfrac{4}{8} = \dfrac{8}{8} =1 $
	\end{center} 
	\vspace{1cm}
	2) A tabela a seguir fornece a função de probabilidade conjunta da v.a. (X, Y).
	\vspace{0.25cm}\\
	\begin{center}
		\begin{tabular}{|c|c|c|c|}\hline
			\multirow{2}{*}{X} & \multicolumn{3}{c|}{Y}\\ \cline{2-4}
			& 1 & 2 & 3\\ \hline
			0 & 0,10 & 0,20 & 0,20\\ \hline
			1 & 0,04 & 0,08 & 0,08\\ \hline
			2 & 0,06 & 0,12 & 0,12\\ \hline
		\end{tabular}
	\end{center}
	\vspace{0.25cm}
	a) (0,5 pontos) Verifique se X e Y são independentes\\
	\begin{center}
		E(X) = $1 * 0,2 + 2 * 0,40 + 3 * 0,4$ = 2,2
		\vspace{0.5cm}\\
		E(Y) = $0 * 0,5 + 1 * 0,20 + 2 * 0,3$ = 0,8
		\vspace{0.5cm}\\
		E(XY) = $0 * 0,5 + 1 * 0,04 + 2 * 0,14 + 3 * 0,08 + 4 * 0,12 + 6 * 0,12$ = 1,76
		\vspace{0.5cm}\\
		E(XY) = E(X) * E(Y)\\
		1,76 = 2,2 * 0,8\\
		\vspace{0.5cm}
		X e Y são independentes
	\end{center}
	\vspace{1cm}
	b) (1 ponto) Encontre a distribuição de X dado Y = 2\\
	\begin{center}
		\begin{tabular}{cc}
			X & $\mathbb{P}$(X = x| Y = 2)\\ \midrule
			0 & $\dfrac{0,20}{0,40} = 0,5$\\ \midrule
			1 & $\dfrac{0,08}{0,40} = 0,2$\\ \midrule
			2 & $\dfrac{0,12}{0,40} = 0,3$\\ \midrule
		\end{tabular}
	\end{center}
	\vspace{1cm}
	c) (1 ponto) Encontre E(X|Y = 2)\\
	\begin{center}
		E(X|Y = 2) = 0 * 0,5 + 1 * 0,2  + 2 * 0,3 = 0,8
	\end{center}
	\vspace{1cm}
	3) Numa caixa encontram-se 5 moedas de ouro e 2 de prata. Serão retiradas 2 moedas com reposição. Considere as variáveis aleatórias:\\
	X = Quantidade de moedas de ouro\\
	Y = Quantidade de moedas de prata\\
	a) (1 ponto) Construa a função de distribuição de probabilidade conjunta de (X, Y).\\
	\begin{center}
		\begin{tikzpicture}
			\tikzstyle{level 1} = [sibling distance=6cm, level distance=2cm]
			\tikzstyle{level 2} = [sibling distance=2cm, level distance=3cm]
			\node{}
				child{
				node{Ou}
					child{
					node{Ou}
					edge from parent
					node[fill=white]{\(\dfrac{4}{6}\)}
					}
					child{
					node{Pr}
					edge from parent
					node[fill=white]{\(\dfrac{2}{6}\)}
					}
				edge from parent
				node[fill=white]{\(\dfrac{5}{7}\)}
				}
				child{
				node{Pr}
					child{
					node{Ou}
					edge from parent
					node[fill=white]{\(\dfrac{5}{6}\)}
					}		
					child{
					node{Pr}
					edge from parent
					node[fill=white]{\(\dfrac{1}{6}\)}
					}
				edge from parent
				node[fill=white]{\(\dfrac{2}{7}\)}		
				};
		\end{tikzpicture}
	\end{center}
	\vspace{1cm}
		\[
	S_{X} =
	\begin{cases}
	(Pr, Pr) = 0\\
	(Pr, Ou); (Ou, Pr) = 1\\
	(Ou, Ou) = 2\\
	\end{cases}
	\]
	\vspace{0.5cm}\\
	\begin{center}
		$\mathbb{P}$(X = 0) = $\dfrac{2}{7} * \dfrac{1}{6} = \dfrac{2}{42}$
		\vspace{0.5cm}\\
		$\mathbb{P}$(X = 1) = $\dfrac{5}{7} * \dfrac{2}{6} + \dfrac{2}{7} * \dfrac{5}{6} = \dfrac{20}{42}$
		\vspace{0.5cm}\\
		$\mathbb{P}$(X = 2) = $\dfrac{5}{7} * \dfrac{4}{6} = \dfrac{20}{42}$
	\end{center}
	\vspace{0.5cm}
	\[
	S_{Y} =
	\begin{cases}
	(Ou, Ou) = 0\\
	(Ou, Pr); (Pr, Ou) = 1\\
	(Pr, Pr) = 2\\
	\end{cases}
	\]
	\vspace{0.5cm}\\
	\begin{center}
		$\mathbb{P}$(Y = 0) = $\dfrac{5}{7} * \dfrac{4}{6} = \dfrac{20}{42}$
		\vspace{0.5cm}\\
		$\mathbb{P}$(Y = 1) = $\dfrac{5}{7} * \dfrac{2}{6} + \dfrac{2}{7} * \dfrac{5}{6} = \dfrac{20}{42}$
		\vspace{0.5cm}\\
		$\mathbb{P}$(Y = 2) = $\dfrac{2}{7} * \dfrac{1}{6} = \dfrac{2}{42}$
	\end{center}
	\vspace{1cm}
	\begin{center}
		\begin{tabular}{ccccc}
			\multirow{2}{*}{Y} & \multicolumn{3}{c}{X} & $\mathbb{P}$(Y = y)\\ \cmidrule{2-4}
			& 0 & 1 & 2 & \\ \midrule
			0 & 0 & 0 & $\dfrac{20}{42}$ & $\dfrac{20}{42}$ \\ \midrule
			1 & 0 & $\dfrac{20}{42}$ & 0 & $\dfrac{20}{42}$\\ \midrule
			2 & $\dfrac{2}{42}$ & 0 & 0 & $\dfrac{2}{42}$\\ \midrule
			$\mathbb{P}$(X = x) & $\dfrac{2}{42}$ & $\dfrac{20}{42}$ & $\dfrac{20}{42}$ & 1\\ \midrule
		\end{tabular}
	\end{center}
	\vspace{1cm}
	b) (0,5 pontos) Verifique se X e Y são independentes.\\
	\begin{center}
			$\mathbb{P}$(X = x) * $\mathbb{P}$(Y = y) = $\mathbb{P}$(X = x $\cap$ Y = y)
			\vspace{0.5cm}\\
			$\mathbb{P}$(X = 0) * $\mathbb{P}$(Y = 0) = $\mathbb{P}$(X = 0 $\cap$ Y = 0)\\
			\vspace{0.25cm}
			$\dfrac{20}{42} * \dfrac{2}{42} \neq 0$\\
			\vspace{1cm}
			X e Y não são independentes
	\end{center}
	\vspace{1cm}
	c) (1 ponto) Encontre a distribuição de probabilidade de Y dado X = 0.\\
	\begin{center}
		\begin{tabular}{cc}
			Y & $\mathbb{P}$(Y = y | X = 0)\\ \midrule
			0 & $\dfrac{0}{\dfrac{2}{42}} = 0$\\ \midrule
			1 & $\dfrac{0}{\dfrac{22}{20}} = 0$\\ \midrule
			2 & $\dfrac{\dfrac{2}{42}}{\dfrac{2}{42}} = 1$\\ \midrule
		\end{tabular}
	\end{center}
	\vspace{1cm}
	4) Considere a seguinte distribuição de probabioidade conjunta (X, Y)
	\vspace{0.25cm}\\
	\begin{center}
		\begin{tabular}{|c|c|c|}\hline
			\multirow{2}{*}{X} & \multicolumn{2}{c|}{Y}\\ \cline{2-3}
			& 0 & -1\\ \hline
			0 & 0,1 & 0,05\\ \hline
			-1 & 0,3 & 0,15\\ \hline
			2 & 0,2 & 0,2\\ \hline
		\end{tabular}
	\end{center}
	\vspace{0.25cm}
	a) (0,5 pontos) Encontre E(X|Y = -1).\\
	\vspace{0.5cm}
	\begin{center}
		\begin{tabular}{cc}
			X & $\mathbb{P}$(X = x | Y = -1)\\ \midrule
			0 & $\dfrac{0,05}{0,22} = 0,2273$\\ \midrule
			-1 & $\dfrac{0,15}{0,22} = 0,6818$\\ \midrule
			2 & $\dfrac{0,02}{0,22} = 0,0900$\\ \midrule
		\end{tabular}
	\end{center}
	\vspace{1cm}
	b) (1 ponto) Encontre a distribuição de probabilidade de W = X - 2Y.\\
	\vspace{0.5cm}
	\begin{center}
		\begin{tabular}{ccc}
			X & Y & W\\ \midrule
			0 & 0 & 0\\ \midrule
			0 & -1 & 2\\ \midrule
			-1 & 0 & -1\\ \midrule
			-1 & -1 & 1\\ \midrule
			2 & 0 & 2\\ \midrule
			2 & -1 & 4\\ \midrule
		\end{tabular}
		\hspace{1cm}
		\begin{tabular}{cc}
			W & $\mathbb{P}$(W = w)\\ \midrule
			-1 & 0,3\\ \midrule
			0 & 0,1\\ \midrule
			1 & 0,15\\ \midrule
			2 & 0,25\\ \midrule
			4 & 0,2\\ \midrule
		\end{tabular}
	\end{center}
	\vspace{1cm}
	c) (1 ponto) Encontre E(W).
	\vspace{0.5cm}
	\begin{center}
		E(W) = -1 * 0,3 + 0 * 0,1 + 1 * 0,15 + 2 * 0,25 + 4 * 0,2\\
		E(W)= -0,3 + 0 +0,1 + 0,5 + 0,8 = 1,1
	\end{center}
	\vspace{1cm}
	\textbf{Desafio para bônus (1 ponto):}
	\vspace{0.25cm}\\
	Foi feito o loteamento de uma área rural em terrenos retangulares. Para cada terreno, seu comprimento e sua largura, ambos em metros, podem ser iguais a 10m, 20m ou 30m. Para simplificar, vamos trabalhar em decâmetros (simbolicamente dam), lembrando que 1dam = 10m. Assim, tanto o comprimento X como a largura Y de um terreno sorteado ao acaso, podem ser iguais a 1, 2 ou 3. A tabela a seguir fornece (apenas parcialmente) a distribuição conjunta de X e de Y, variáveis aleatórias supostas independentes:\\
	\vspace{0.25cm}\\
	\begin{center}
		\begin{tabular}{|c|c|c|c|c|}\hline
			\multirow{2}{*}{X} & \multicolumn{3}{c|}{Y} & \multirow{2}{*}{$\mathbb{P}$(X)}\\ \cline{2-4}
			& 1 & 2 & 3 & \\ \hline
			1 & 0,35 & & 0,14 & \\ \hline
			2 & & & & 0,20\\ \hline
			3 & 0,05 & & & \\ \hline
			$\mathbb{P}$(Y) & & 0,30 & & \\ \hline	
		\end{tabular}
	\end{center}
	\vspace{0.25cm}
	Considere as v.a.'s V = 2X + 2Y, o perímetro do terreno e W = XY, a área do terreno. Calcule:\\
	a) O valor de cada probabilidade, conjunta ou marginal, omitido na tabela acima.\\
	\vspace{0.5cm}
	\begin{center}
		\begin{tabular}{|c|c|c|c|c|}\hline
			\multirow{2}{*}{X} & \multicolumn{3}{c|}{Y} & \multirow{2}{*}{$\mathbb{P}$(X)}\\ \cline{2-4}
			& 1 & 2 & 3 & \\ \hline
			1 & 0,35 & a & 0,14 & b\\ \hline
			2 & c & d & e & 0,20\\ \hline
			3 & 0,05 & f & g & h \\ \hline
			$\mathbb{P}$(Y) & i & 0,30 & j & k\\ \hline	
		\end{tabular}
		\vspace{1cm}\\
		0,35 + c + 0,05 = i
		\vspace{0.25cm}\\
		0,35 + (0,20 * i) + 0,05 = i
		\vspace{0.25cm}\\
		0,40 = 0,8i
		\vspace{0.25cm}\\
		i = 0,5
		\vspace{1cm}\\
		0,20 * i = c 
		\vspace{0.25cm}\\
		0,20 * 0,5 = c
		\vspace{0.25cm}\\
		c = 0,1
		\vspace{1cm}\\
		k = 1
		\vspace{1cm}\\
		i + 0,30 + j = k
		\vspace{0.25cm}\\
		0,5 + 0,30 + j = 1
		\vspace{0.25cm}\\
		j = 0,2
		\vspace{1cm}\\
		b * j = 0,14
		\vspace{0.25cm}\\
		0,2 * b = 0,14
		\vspace{0.25cm}\\
		b = 0,7
		\vspace{1cm}\\
		b + 0,2 + h = k
		\vspace{0.25cm}\\
		0,7 + 0,2 + h = 1
		\vspace{0.25cm}\\
		h = 0,1
		\vspace{1cm}\\
		0,35 + a + 0,14 = b
		\vspace{0.25cm}\\
		0,35 + a + 0,14 = 0,7
		\vspace{0.25cm}\\
		a = 0,21
		\vspace{1cm}\\
		d = 0,30 * 0,20
		\vspace{0.25cm}\\ 
		d = 0,06
		\vspace{1cm}\\
		f = 0,30 * h
		\vspace{0.25cm}\\
		f = 0,30 * 0,1
		\vspace{0.25cm}\\
		f = 0,03
		\vspace{1cm}\\
		e = 0,20 * j
		\vspace{0.25cm}\\
		e = 0,20 * 0,2
		\vspace{0.25cm}\\
		e = 0,04
		\vspace{1cm}\\
		g = h * j
		\vspace{0.25cm}\\
		g = 0,1 * 0,2
		\vspace{0.25cm}\\
		g = 0,02
		\vspace{1cm}\\
		\begin{tabular}{|c|c|c|c|c|}\hline
			\multirow{2}{*}{X} & \multicolumn{3}{c|}{Y} & \multirow{2}{*}{$\mathbb{P}$(X)}\\ \cline{2-4}
			& 1 & 2 & 3 & \\ \hline
			1 & 0,35 & 0,21 & 0,14 & 0,70\\ \hline
			2 & 0,10 & 0,06 & 0,04 & 0,20\\ \hline
			3 & 0,05 & 0,03 & 0,02 & 0,10 \\ \hline
			$\mathbb{P}$(Y) & 0,5 & 0,30 & 0,2 & 1\\ \hline	
		\end{tabular}
	\end{center}
	\vspace{1.5cm}
	b) A probabilidade condicional de que a área seja igual a 4dam$^2$, dado que o perímetro é igual a 8dam.\\
	\begin{center}
		\begin{tabular}{ccccc}
			X & Y & V & W & $\mathbb{P}$(X = x $\cap$ Y = y)\\ \midrule
			1 & 1 & 4 & 1 & 0,35 \\ \midrule
			1 & 2 & 6 & 2 & 0,10 \\ \midrule
			1 & 3 & 8 & 3 & 0,05 \\ \midrule
			2 & 1 & 6 & 2 & 0,21 \\ \midrule
			2 & 2 & 8 & 4 & 0,06\\ \midrule
			2 & 3 & 10 & 6 & 0,03\\ \midrule
			3 & 1 & 8 & 3 & 0,14\\ \midrule
			3 & 2 & 10 & 6 & 0,04\\ \midrule
			3 & 3 & 12 & 9 & 0,02\\ \midrule		
		\end{tabular}
		\vspace{1cm}\\
		\begin{tabular}{|c|c|c|c|c|c|c|c|}\hline
			\multirow{2}{*}{V} & \multicolumn{6}{c|}{W} & \multirow{2}{*}{$\mathbb{P}$(V)}\\ \cline{2-7}
			& 1 & 2 & 3 & 4 & 6 & 9 & \\ \hline
			4 & 0,35 & 0 & 0 & 0 & 0 & 0 & 0,35\\ \hline
			6 & 0 & 0,31 & 0 & 0 & 0 & 0 & 0,31\\ \hline
			8 & 0 & 0 & 0,19 & 0,06 & 0 & 0 & 0,19\\ \hline
			10 & 0 & 0 & 0 & 0 & 0,07 & 0 & 0,07\\ \hline
			12 & 0 & 0 & 0 & 0 & 0 & 0,02 & 0,02\\ \hline
			$\mathbb{P}$(W) & 0,35 & 0,31 & 0,19 & 0,06 & 0,07 & 0,02 & 1\\ \hline
		\end{tabular}
		\vspace{1cm}\\
		$\mathbb{P}$(W = 4 | V = 8) = $\dfrac{\mathbb{P}(W = 4) * \mathbb{P}(V = 8)}{\mathbb{P}(V = 8)} = \mathbb{P}(W = 4) = 0,06$  
	\end{center}
	\vspace{2cm}
	Formulário:\\
	\vspace{0.25cm}\\
	E(X) = $\sum_{i=1}^{n}$ x$_{i}$ * $\mathbb{P}$(X = x$_{i}$) 
	\vspace{0.25cm}\\
	$\mathbb{P}$(X = x$_{i}$ | Y = y$_{j}$) = $\dfrac{\mathbb{P}(X = x_{i}, Y = y_{j})}{\mathbb{P}(Y = y_{j})}$
	\vspace{0.25cm}\\
	Se X e Y são independentes, então: $\mathbb{P}(X = x_{i}, Y = y_{j}) = \mathbb{P}(X = x_{i}) * \mathbb{P}(X = x_{j})$
	\vspace{0.25cm}\\
	E(X | Y = y$_{j}$) = $\sum_{i=1}^{n}$ x$_{i}$ * $\mathbb{P}$(X = x$_{i}$ | Y = y$_{j}$)
\end{document}