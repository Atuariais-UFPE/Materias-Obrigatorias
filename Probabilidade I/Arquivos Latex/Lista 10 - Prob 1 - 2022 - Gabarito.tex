\documentclass[12pt,a4paper]{article}
\usepackage[utf8]{inputenc}
\usepackage[T1]{fontenc}
\usepackage{amsmath}
\usepackage{amsfonts}
\usepackage{amssymb}
\usepackage{graphicx}
\usepackage[width=0.00cm, left=3.00cm, right=2.00cm, top=3.00cm, bottom=2.00cm]{geometry}
\title{Lista 10}
\date{}
\begin{document}
	\maketitle
	\begin{center}
		\textbf{Curso de Ciências atuariais}\\
		\textbf{Disciplina Probabilidade 1- Professora Cristina}\\
		\textbf{02/09/2022 - Exercícios distribuição de Poisson}
	\end{center}
	1) Um departamento de polícia recebe em média 5 solicitações por hora. Qual a probabilidade de	receber 2 solicitações em duas horas selecionada aleatoriamente?
	\vspace{0.5cm}\\
	X = O departamento de polícia receber solicitações
	\begin{center}
		\vspace{0.5cm}
		$\lambda$ = 5 ligações/ hora * 2 = 10 ligações/2 horas\\
		\vspace{1cm}
		$\mathbb{P}$(X = 2) = $\dfrac{e^{-10} * 10^{2}}{2!}$ = $\dfrac{100e^{-10}}{2}$ = $\dfrac{0,00454}{2}$ = 0,00227 
	\end{center}
	\vspace{1cm}
	2) A experiência passada indica que um número médio de 6 clientes por hora para para colocar gasolina numa bomba.\\
	a) Qual é a probabilidade de 3 clientes pararem qualquer hora?
	\vspace{0.5cm}\\
	X = Clientes pararem na bomba para abastecer
	\begin{center}
		\vspace{0.5cm}
		$\mathbb{P}$(X = 3) = $\dfrac{e^{-6} * 6^{3}}{3!}$ = $\dfrac{216e^{-6}}{6}$ = $\dfrac{0,53541}{6}$ = 0,08923 
	\end{center}
	\vspace{1cm}
	b) Qual é a probabilidade de 3 clientes ou menos pararem em qualquer hora?
	\begin{center}
		\vspace{0.5cm}
		$\mathbb{P}$(X $\leq$ 3) = $\mathbb{P}$(X = 0) + $\mathbb{P}$(X = 1) + $\mathbb{P}$(X = 2) + $\mathbb{P}$(X = 3)
		\vspace{0.5cm}\\
		$\mathbb{P}$(X $\leq$ 3) = $\dfrac{e^{-6} * 6^{0}}{0!}$ + $\dfrac{e^{-6} * 6^{1}}{1!}$ + $\dfrac{e^{-6} * 6^{2}}{2!}$ + $\dfrac{e^{-6} * 6^{3}}{3!}$
		\vspace{0.5cm}\\
		$\mathbb{P}$(X $\leq$ 3) =  $\dfrac{e^{-6}}{1}$ + $\dfrac{6e^{-6}}{1}$ + $\dfrac{36e^{-6}}{2}$ + $\dfrac{216e^{-6}}{6}$
		\vspace{0.5cm}\\
		$\mathbb{P}$(X $\leq$ 3) = ${61e^{-6}}$ = 0,1512
	\end{center}
	\vspace{1cm}
	c) Qual é o valor esperado (média) e o desvio padrão para esta distribuição?
	\begin{center}
		\vspace{0.5cm}
		E(X) = V(X) = $\lambda$ = 6
		\vspace{1cm}\\
		$\sigma$ = $\sqrt{V(X)}$ = 2,4495
	\end{center}
	\vspace{1cm}
	3) A experiência passada mostra que 1\% das lâmpadas incandescentes produzidas numa fábrica são defeituosas. Encontre a probabilidade de mais que uma lâmpada numa amostra aleatória de 30 lâmpadas sejam defeituosas, usando:\\
	a) A distribuição Binomial;
	\vspace{0.5cm}\\
	X = A lâmpada seja defeituosa
	\begin{center}
		\vspace{0.5cm}
		$\mathbb{P}$(X $\geq$ 1) = 1 - $\mathbb{P}$(X < 1) =  1 - $\mathbb{P}$(X = 0)
		\vspace{0.5cm}\\
		$\mathbb{P}$(X $\geq$ 1) = 1 - $\left[\dbinom{30}{0} * 0,1^{0} * 0,99^{30}\right]$ = 1 - 0,7397 = 0,2603
	\end{center}
	\vspace{1cm}
	b) A distribuição de Poisson
	\begin{center}
		\vspace{0.5cm}
		$\lambda$ = np = 30 * 0,01 = 0,3 lâmpada defeituosa/hora
		\vspace{1cm}\\
		$\mathbb{P}$(X $\geq$ 1) = 1 - $\left[\dfrac{e^{-0,3} * 0,3^{0}}{0!}\right]$ = 1 - 0,7408 = 0,2592`
	\end{center}
	\vspace{1cm}
	4) Um  processo  de  produção  produz  10 itens  defeituosos  por  hora.  Encontre a probabilidade  que  4  ou menos itens sejam defeituosos numa retirada aleatória por hora usando a distribuição de Poisson.
	\vspace{0.5cm}\\
	X = Obter um item defeituoso
	\begin{center}
		\vspace{0.5cm}
		$\mathbb{P}$(X $\leq$ 4) = $\mathbb{P}$(X = 0) + $\mathbb{P}$(X = 1) + $\mathbb{P}$(X = 2) + $\mathbb{P}$(X = 3) + $\mathbb{P}$(X = 4)
		\vspace{0.5cm}\\
		$\mathbb{P}$(X $\leq$ 4) = $\dfrac{e^{-10} * 10^{0}}{0!}$ + $\dfrac{e^{-10} * 10^{1}}{1!}$ + $\dfrac{e^{-10} * 10^{2}}{2!}$ + $\dfrac{e^{-10} * 10^{3}}{3!}$ +$\dfrac{e^{-10} * 10^{4}}{4!}$
		\vspace{0.5cm}\\
		$\mathbb{P}$(X $\leq$ 4) = $\dfrac{e^{-10}}{1}$ + $\dfrac{10e^{-10}}{1}$ + $\dfrac{100e^{-10}}{2}$ + $\dfrac{1.000e^{-10}}{6}$ + $\dfrac{10.000e^{-10}}{24}$= $\dfrac{1.933e^{-10}}{3}$
		\vspace{0.5cm}\\
		$\mathbb{P}$(X $\leq$ 4) = 0,08776
	\end{center}
	\vspace{1cm}
	5) O número de petroleiros que chegam a uma refinaria em cada dia ocorre segundo uma distribuição de Poisson, com parâmetro 2. As atuais instalações podem atender, no máximo, a três petroleiros por dia. Se mais de três aportarem num dia, o excesso é enviado a outro porto.\\
	a) Em um dia, qual a probabilidade de se enviar petroleiros para outro porto?
	\vspace{0.5cm}\\
	X = Chegar um navio petroleiro
	\begin{center}
		\vspace{0.5cm}
		$\mathbb{P}$(X $\geq$ 4) = 1 - $\mathbb{P}$(X < 4) = 1 - [$\mathbb{P}$(X = 0) + $\mathbb{P}$(X = 1) + $\mathbb{P}$(X = 2) + $\mathbb{P}$(X = 3)]
		\vspace{0.5cm}\\
		$\mathbb{P}$(X $\geq$ 4) = 1 - $\left[\dfrac{e^{-2} * 2^0}{0!} + \dfrac{e^{-2} * 2^1}{1!} + \dfrac{e^{-2} * 2^2}{2!} + \dfrac{e^{-2} * 2^3}{3!}\right]$
		\vspace{0.5cm}\\
		$\mathbb{P}$(X $\geq$ 4) = 1 - $\left[\dfrac{e^{-2}}{1} + \dfrac{2e^{-2}}{1} + \dfrac{4e^{-2}}{2} + \dfrac{8e^{-2}}{6}\right]$ = 1 - $\dfrac{19e^{-2}}{3}$
		\vspace{0.5cm}\\
		$\mathbb{P}$(X $\geq$ 4) = 1 - 0,85712 = 0,14288
	\end{center}
	\vspace{1cm}
	b) Qual o número médio de petroleiros que chegam por dia?
	\vspace{0.5cm}\\
	\begin{center}
		E(X) = $\lambda$ = 2
	\end{center}
	\vspace{1cm}
	6) Um vendedor de seguros vende em média 3 apólices por semana (7 dias). O número de	apólices vendidas pode ser modelada por uma distribuição de Poisson.\\
	a) Calcule a probabilidade de que ele venda 2 ou mais apólices numa dada semana
	\vspace{0.5cm}\\
	X = Vender uma apólice de seguro
	\begin{center}
		\vspace{0.5cm}
		$\mathbb{P}$(X $\geq$ 2) = 1 - $\mathbb{P}$(X < 2) = 1 - [$\mathbb{P}$(X = 0) + $\mathbb{P}$(X = 1)]
		\vspace{0.5cm}\\
		$\mathbb{P}$(X $\geq$ 2) = 1 - $\left[\dfrac{e^{-3} * 3^0}{0!} + \dfrac{e^{-3} * 3^1}{1!}\right]$ = 1 - $\left[\dfrac{e^{-3}}{1} + \dfrac{3e^{-3}}{1}\right]$ = 1 - $\dfrac{4e^{-3}}{1}$
		\vspace{0.5cm}\\
		$\mathbb{P}$(X $\geq$ 2) = 1 - 0,19915 = 0,80085
	\end{center}
	\vspace{1cm}
	b) Foram escolhidas 4 semanas aleatoriamente, de maneira que se possa
	supor independência das vendas entre as semanas. Deseja-se saber a
	probabilidade de em exatamente 3 semanas, entre as 4 escolhidas, terem
	sido vendidas 2 ou mais apólices
	\vspace{0.5cm}\\
	Y = Bater a meta semanal de venda apólices de seguro
	\begin{center}
		\vspace{0.5cm}
		$\mathbb{P}$(X = 3) = $\dbinom{4}{3} * 0,80085^{3} * 0,19915^{1}$ 
		\vspace{0.5cm}\\
		$\mathbb{P}$(X = 3) = $\dfrac{4!}{3!1!} * 0,51364 * 0,19915$ = 0,40916
	\end{center}
	\vspace{1cm}
	7) Uma fonte radioativa é observada durante intervalos de tempo, cada um de 1 segundo de duração. O número de partículas emitidas (X), durante cada intervalo, tem uma Distribuição de Poisson com parâmetro
	0,5 partículas/segundo.\\
	a) Qual a probabilidade – para um intervalo de tempo – de que 5 ou mais partículas sejam emitidas?
	\vspace{0.5cm}\\
	X = Observar a emissão de partículas
	\begin{center}
		\vspace{0.5cm}
		$\mathbb{P}$(X $\geq$ 5) = 1 - $\mathbb{P}$(X < 5) = 1 - [$\mathbb{P}$(X = 0) + $\mathbb{P}$(X = 1) + $\mathbb{P}$(X = 2) + $\mathbb{P}$(X = 2)]
		\vspace{0.5cm}\\
		$\mathbb{P}$(X $\geq$ 5) = 1 - $\left[\dfrac{e^{-0,5} * 0,5^0}{0!} + \dfrac{e^{-0,5} * 0,5^1}{1!} + \dfrac{e^{-0,5} * 0,5^2}{2!} + \dfrac{e^{-0,5} * 0,5^3}{3!}\right]$
		\vspace{0.5cm}\\
		$\mathbb{P}$(X $\geq$ 5) = 1 - $\left[\dfrac{e^{-0,5}}{1} + \dfrac{0,5e^{-0,5}}{1} + \dfrac{0,25e^{-0,5}}{2} + \dfrac{0,125e^{-0,5}}{6}\right]$
		\vspace{0.5cm}\\
		$\mathbb{P}$(X $\geq$ 5) = 1 - $\dfrac{5,3125e^{-0,5}}{3}$ = 1 - 0,53703 = 0,46297
	\end{center}
	\vspace{1cm}
	b) Qual a probabilidade de que, em pelo menos 1 dos 6 intervalos de
	tempo, 5 ou mais partículas sejam emitidas?
	\vspace{0.5cm}\\
	Y = Observar mais que 5 partículas serem emitidas no intervalo de tempo
	\begin{center}
		\vspace{0.5cm}
		$\mathbb{P}$(X $\geq$ 1) = 1 -	$\mathbb{P}$(X < 1) = 1 -	$\mathbb{P}$(X = 0)
		\vspace{0.5cm}\\
		$\mathbb{P}$(X $\geq$ 1) = 1 - $\left[\dbinom{6}{0} * 0,46297^{0} * 0,53703^{6}\right]$ 
		\vspace{0.5cm}\\
		$\mathbb{P}$(X $\geq$ 1) = 1 - $\left[\dfrac{6!}{0!6!} * 1 * 0,02399\right]$ = 0,97601
	\end{center}
	\vspace{1cm}
	8) O pessoal de controle de qualidade de uma empresa de asfalto, afirma que uma das rodovias que conecta o Rio de Janeiro com Minas gerais apresenta em média um buraco a cada 50 km. Admitindo que a distribuição do número de buracos a	cada 50 km é modelado por uma distribuição de Poisson, calcule as	probabilidades:\\
	a) De que não exista nenhum buraco em 30 km.
	\vspace{0.5cm}\\
	X = Encontrar um buraco na rodovia
	\begin{center}
		\vspace{0.5cm}
		$\lambda$ = $\dfrac{1 * 30}{50}$ = 0,6 buraco/trecho da rodovia
		\vspace{1cm}\\
		$\mathbb{P}$(X = 0) = $\dfrac{e^{-0,6} * 0,6^{0}}{0!}$ = $\dfrac{e^{-0,6}}{1}$ = 0,54881
	\end{center}
	\vspace{1cm}
	b) De ocorrerem no máximo dois buracos em 125 km.
	\begin{center}
		\vspace{0.5cm}
		$\lambda$ = $\dfrac{1 * 125}{50}$ = 2,5 buraco/trecho da rodovia
		\vspace{1cm}\\
		$\mathbb{P}$(X $\leq$ 2) = $\mathbb{P}$(X = 0) + $\mathbb{P}$(X = 1) + $\mathbb{P}$(X = 2)
		\vspace{1cm}\\
		$\mathbb{P}$(X $\leq$ 2) = $\dfrac{e^{-2,5} * 2,5^{0}}{0!}$ + $\dfrac{e^{-2,5} * 2,5^{1}}{1!}$ + $\dfrac{e^{-2,5} * 2,5^{2}}{2!}$
		\vspace{1cm}\\
		$\mathbb{P}$(X $\leq$ 2) = $\dfrac{e^{-2,5}}{1}$ + $\dfrac{2,5e^{-2,5}}{1}$ + $\dfrac{6,25e^{-2,5}}{2}$
		\vspace{1cm}\\
		$\mathbb{P}$(X $\leq$ 2) = $\dfrac{6,625e^{-2,5}}{1}$ = 0,54381
	\end{center}
	c) De ocorrer pelo menos um buraco em 100 km.
	\begin{center}
		\vspace{0.5cm}
		$\lambda$ = $\dfrac{1 * 100}{50}$ = 2 buraco/trecho da rodovia
		\vspace{1cm}\\
		$\mathbb{P}$(X $\geq$ 1) = 1 - $\mathbb{P}$(X < 1) = 1 - $\mathbb{P}$(X = 0)
		\vspace{1cm}\\
		$\mathbb{P}$(X $\geq$ 1) = 1 - $\dfrac{e^{-2} * 2^0}{0!}$ = 1 - $\dfrac{e^{-2}}{1}$ = 1 - 0,13533 = 0,86467
	\end{center}
\end{document}